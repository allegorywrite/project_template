\section{結論}

本論文では,SE(3)上における協調自己位置推定のための視野共有を保証する分散型制御バリア関数(CBF)手法を提案した.提案手法の主な貢献は以下の通りである.

第一に,SE(3)における共有視野保証を実現した.従来のステレオ視やリーダ–フォロワ形式による視野制御は,ロボット間の相対配置を幾何学的に制約するに留まっていたが,本手法は3次元空間におけるエージェント全体の姿勢・位置を統合的に制御する枠組みを提供した.

第二に,特徴点に基づく確率的可視性制約とCBFの適用を行った.各エージェントが観測する特徴点に基づき,その可視性を確率的に評価した上で,CBFに組み込み,常時高い確率で共有視野が確保されるよう制御入力を設計した.これにより,センサノイズ等による不確実性下でも安全な視野維持が可能となった.

第三に,非ホロノミックなドローンダイナミクスへの対応と分散最適化を実現した.機体の並進および回転運動を同時に考慮するSE(3)上の非ホロノミックドローンモデルに対して,高次制約も扱える高次制御バリア関数(HOCBF)を導入し,各エージェントが局所的な情報交換を通じて分散最適化アルゴリズム(PDMM等)により制御解を求める枠組みを提案した.これにより,リアルタイム性とスケーラビリティの両立を実現した.

シミュレーション結果から,提案手法は単一特徴点および複数特徴点を追従する場合において,共有視野を保証できることが確認された.また,分散実装においても中央集権的な実装と同等の性能を維持しつつ,計算効率が向上することが示された.

今後の課題としては,以下の点が挙げられる.

第一に,実機実験による検証が必要である.シミュレーションでは理想的な条件下で検証を行ったが,実環境では通信遅延やセンサノイズ,モデル誤差などの影響を受ける.これらの要因を考慮した実機実験を通じて,提案手法の有効性をさらに検証する必要がある.

第二に,自己位置推定アルゴリズムとの統合が挙げられる.本研究では,共有視野を保証する制御手法に焦点を当てたが,実際の協調自己位置推定では,視野共有と自己位置推定を統合的に扱う必要がある.特に,特徴点マッチングの結果を制御にフィードバックする枠組みの構築が重要である.

第三に,動的環境への対応が課題である.本研究では,静的な特徴点を対象としたが,実環境では特徴点が移動したり,新たな特徴点が現れたりする.このような動的環境に対応するため,オンラインでの特徴点検出と追跡,および制御則の適応的な更新が必要である.

第四に,より大規模なマルチエージェントシステムへの拡張が考えられる.本研究では,少数のエージェントを対象としたが,実際のアプリケーションでは多数のエージェントが協調する場合がある.このような大規模システムに対して,通信トポロジーの最適化や計算負荷の分散化など,さらなる改良が必要である.

以上の課題に取り組むことで,提案手法の実用性をさらに高め,マルチロボットシステムにおける協調自己位置推定の精度向上に貢献することが期待される.
