%%%%%%%%%%%%%%%%%%%%%%%%%%%%%%%%%%%%%%%%%%%%%%%%%%%%
\documentclass[a4paper,fleqn,10pt]{SICE_ISCS}
%\usepackage{url}
\usepackage{ascmac}
\usepackage{amssymb}
%\usepackage{amsmath}
%\usepackage{hyperref}
%\usepackage{lmodern}
\usepackage{breqn}
\usepackage{bm}
\usepackage{comment}
\usepackage{pdfpages}
\usepackage{algorithm}
\usepackage{algorithmic}

\makeatletter
\@ifundefined{theorem}{%
  \newtheorem{theorem}{Theorem}
}{}
\@ifundefined{definition}{%
  \newtheorem{definition}{Definition}
}{}
\makeatother

%\usepackage{HERE}
%\usepackage[version=3]{mhchem}%%化学式
%\usepackage{siunitx}
% Packages for Japanese language support
%\usepackage{CJKutf8}
%\usepackage{otf}
\usepackage[dvipdfmx]{hyperref}
%\usepackage{pxjahyper}
\usepackage{cite}
\usepackage{ulem} % Package for strikethrough
\usepackage{caption}
\captionsetup{font=small} % キャプションのフォントサイズを小さく設定
\usepackage{float} % figureの位置を固定するためのパッケージ
\DeclareGraphicsExtensions{.eps,.pdf,.png,.jpg}
\newcommand{\Tabref}[1]{{Table~\ref{#1}}}
\newcommand{\Eqref}[1]{式~(\ref{#1})}
\newcommand{\Figref}[1]{{Fig.~\ref{#1}}}
\newcommand{\blue}[1]{\textcolor{blue}{#1}}
\newcommand{\red}[1]{\textcolor{red}{#1}}
\title{SE(3)上における協調自己位置推定のための視野共有を保証する分散型CBF}

\author{著者名${}^{1\dagger}$}
% The dagger symbol indicates the presenter.
\speaker{著者名}

\affils{${}^{1}$所属機関, 都道府県, 国\\
(Tel: +81-xx-xxxx-xxxx; E-mail: example@example.com)\\
}

\abstract{%
本論文では,SE(3)上における協調自己位置推定のための視野共有を保証する分散型制御バリア関数(CBF)を提案する.マルチエージェントVSLAMでは,エージェント間で特徴点のマッチングを行うために視野の重複が必要であるが,従来手法では視野共有を積極的に保証する制御則は十分に検討されていなかった.本研究では,エージェント位置を$T_i=(p_i, R_i)\in \mathrm{SE}(3)$,環境内の特徴点を$q_l\in \mathcal{L}$として,視野内条件に基づく安全集合を定義し,CBFを設計する.さらに,複数の特徴点を追従する場合には確率的アプローチを導入し,複数エージェントの共通特徴点追従のための分散型実装を提案する.また,実機への適用を考慮し,ドローンの二次系ダイナミクスに対応する高次CBF(HOCBF)を設計する.シミュレーション結果から,提案手法は安全制約を満たしながら目標位置への追従が可能であり,分散実装により計算効率とスケーラビリティが向上することが示された.本研究の成果は,ドローンの協調自己位置推定だけでなく,様々なマルチロボットタスクにおいて視野共有を保証するための基盤技術として応用可能である.
}

\keywords{%
制御バリア関数,視野共有,マルチエージェントシステム,SE(3),分散最適化,高次制御バリア関数
}


\begin{document}

% タイトルを手動で設定(シングルカラム用)
\begin{center}
\Large\bf SE(3)上における協調自己位置推定のための視野共有を保証する分散型CBF\par
\vskip1.0em
\large 著者名${}^{1\dagger}$\par
\vskip0.5em
\normalsize ${}^{1}$所属機関, 都道府県, 国\\
(Tel: +81-xx-xxxx-xxxx; E-mail: example@example.com)\par
\vskip1.0em
\end{center}

\noindent{\normalsize{\bf {Abstract:} }本論文では,SE(3)上における協調自己位置推定のための視野共有を保証する分散型制御バリア関数(CBF)を提案する.マルチエージェントVSLAMでは,エージェント間で特徴点のマッチングを行うために視野の重複が必要であるが,従来手法では視野共有を積極的に保証する制御則は十分に検討されていなかった.本研究では,エージェント位置を$T_i=(p_i, R_i)\in \mathrm{SE}(3)$,環境内の特徴点を$q_l\in \mathcal{L}$として,視野内条件に基づく安全集合を定義し,CBFを設計する.さらに,複数の特徴点を追従する場合には確率的アプローチを導入し,複数エージェントの共通特徴点追従のための分散型実装を提案する.また,実機への適用を考慮し,ドローンの二次系ダイナミクスに対応する高次CBF(HOCBF)を設計する.シミュレーション結果から,提案手法は安全制約を満たしながら目標位置への追従が可能であり,分散実装により計算効率とスケーラビリティが向上することが示された.本研究の成果は,ドローンの協調自己位置推定だけでなく,様々なマルチロボットタスクにおいて視野共有を保証するための基盤技術として応用可能である.\par}
\noindent{\vskip 1em \normalsize {\bf {Keywords:} }制御バリア関数,視野共有,マルチエージェントシステム,SE(3),分散最適化,高次制御バリア関数\par}

\vskip1.4em

%-----------------------------------------------------------------------

\section{序論}

\subsection{共有視野保証の重要性と背景}

マルチロボットシステムでは,各ロボットがセンサ情報や視界を共有することにより,監視・捜索・協調輸送などのタスクを効率的に遂行できることが知られている.
特にカメラを用いた視覚協調の場合,各ロボットの共有視野(Common Field-of-View, CoFoV)が不可欠である.例えば,複数ドローンが異なる角度から同一対象を観測できることや,通信の見通し線(LOS)維持が求められる\cite{Panagou2017}.

マルチエージェントVSLAM(Visual Simultaneous Localization and Mapping)においては,各エージェントが撮影した画像から抽出される局所特徴に基づき自己局在を行い,エージェント間で特徴マッチングにより相対位置を推定する.従来のORBやSIFTに代わり,SuperPointのような学習型局所特徴検出・記述器\cite{DeTone2018}や,NetVLADによるグローバル記述子\cite{Arandjelovic2016}が高いロバスト性を示し,ループ検出やキーフレームマッチングに貢献している.

画像特徴量のマッチングを成功させるためには,各エージェントが共通のランドマークを観測できるよう,カメラの視野錐台の幾何学的重複が必要である.視野重複が存在すれば,エージェント間でのループ検出が可能となり,その結果として各ロボットの地図統合が実現される\cite{Zhang2022}.しかし,ロボットは視野角が限られているため,視界共有を保証するための制御技術の開発が急務である.

\red{モチベーション:CoVINSではsuperpoint特徴量などの画像特徴量のマッチングにより、自己位置推定に関する最適化問題のfactorを得ている。特徴量をマッチングするにはエージェントの視野錐台が重なっている必要がある。single agent問題に関してはstereo cameraの相対位置をconstかつgivenとして最適化問題に組み込み、multi agent問題に関しては視野錐台のoverrideは不可知であるために画像全体特徴量の類似度の一致などによってpassiveなevent trigger型としてアルゴリズムが構築されている。しかし、agent1とagent2が視野錐台を交差させ続けるための制御則(CBF等)に基づいてactive perceptionを行う場合、multi agentの自己位置推定においてinter-agentな特徴量のマッチング及び推定問題はエージェントをまたいだカメラ間の相対位置をgiven, もしくは最適化すべき双対変数として複数のエージェント位置を同時最適化できるはずである。さらに、active perceptionの枠組みで考えれば、CBFを用いた最適制御問題と自己位置推定問題も同一の目的関数の最小化問題として扱うことができるはずである。  
上記の仮定から、本セミナーではその手始めとして,代表的なCoVINSであるUAVを対象として、視野錐台を交差させ続けるための分散型CBF手法を提案する。}

\subsection{既存研究の課題}

従来はポテンシャル場やMPC(Model Predictive Control)を用いた手法が提案され,局所的な視界制約下での隊列制御やセンサグラフの連結性維持に取り組まれてきた\cite{Sabattini2013}.一方で,制御バリア関数(Control Barrier Function, CBF)を用いた手法は,制約違反を防ぎながらリアルタイムに最適な制御入力を計算できるため,有力な候補となる\cite{Capelli2020}.

しかし,現状の協調SLAMは,キーフレームの受動的な共有と画像類似度評価に依存しており,視野重複が偶発的に発生しなければマップ統合が成立しないリスクがある.また,多数のロボットを対象とする場合,中央集権的な制御は通信負荷や計算量の面で現実的ではなく,各エージェントが局所的に計算し,限定的な通信で連携する分散アルゴリズムの設計が必要である.

従来の視野維持手法は,対象が視野内にあるか否かを決定論的に評価するに留まっていたが,センサの観測不確実性を十分に取り入れていなかった\cite{Panagou2012}.また,従来の多くの視野制約付き制御手法は,単純な動力学モデル(例:Dubins車両やクアッドロータの水平姿勢維持)に基づいており,非ホロノミックなダイナミクスを明示的に考慮していなかった\cite{Dias2016}.

\subsection{本研究の貢献}

本研究では,SE(3)上における協調自己位置推定のための視野共有を保証する分散型CBF手法を提案する.本研究の主な貢献は以下の通りである.

\begin{itemize}
\item SE(3)における共有視野保証を実現する.従来のステレオ視やリーダーフォロワ形式による視野制御は,ロボット間の相対配置を幾何学的に制約するに留まっていたが,本手法は3次元空間におけるエージェント全体の姿勢・位置を統合的に制御する枠組みを提供する.

\item 特徴点に基づく確率的可視性制約とCBFの適用を行う.本研究では,各エージェントが観測する特徴点に基づき,その可視性を確率的に評価した上で,CBFに組み込み,常時高い確率で共有視野が確保されるよう制御入力を設計する.これにより,センサノイズ等による不確実性下でも安全な視野維持が可能となる.

\item 非ホロノミックなドローンダイナミクスへの対応と分散最適化を実現する.本研究では,機体の並進および回転運動を同時に考慮するSE(3)上の非ホロノミックドローンモデルに対して,高次制約も扱える高次制御バリア関数(HOCBF)を導入し,各エージェントが局所的な情報交換を通じて分散最適化アルゴリズム(PDMM等)により制御解を求める枠組みを提案する\cite{Lv2024}.これにより,リアルタイム性とスケーラビリティの両立を実現する.
\end{itemize}

\subsection{論文構成}

本論文の構成は以下の通りである.第2章ではCBFの基本概念と高次CBFについて説明する.第3章では問題設定とシステムモデルについて述べる.第4章では共有視野のためのCBFを提案し,第5章では二次系システムのための高次CBFを導入する.第6章ではシミュレーション結果を示し,第7章で結論と今後の課題について述べる.

\begin{figure}[htbp]
\centering
\includegraphics[width=0.8\linewidth]{fig/progress.png}
\caption{第一回セミナーの内容と本研究の貢献点}
\label{fig:progress}
\end{figure}
\section{Preliminary: Control Barrier Function}

本章では,本研究の基礎となる制御バリア関数(Control Barrier Function, CBF)の概念について説明する.

\subsection{制御バリア関数の基本概念}

制御バリア関数は,システム状態がある安全集合内に留まることを保証するためのLyapunov関数に類似した概念である.もともとはバリア関数やバリア証明の枠組みに基づいており,制御Lyapunov関数(CLF)と統合した安全・安定制御のための二次計画問題(QP)による実装手法が提案されている\cite{Ames2017}.

制御対象のシステムを以下のように表す.
\begin{equation}
\begin{aligned}
\dot{x} = f(x) + g(x)u
\label{eq:system}
\end{aligned}
\end{equation}
ここで,$x \in \mathbb{R}^n$は状態,$u \in \mathbb{R}^m$は制御入力,$f: \mathbb{R}^n \rightarrow \mathbb{R}^n$および$g: \mathbb{R}^n \rightarrow \mathbb{R}^{n \times m}$は局所Lipschitz連続な関数である.

安全集合$\mathcal{C}$を以下のように定義する.
\begin{equation}
\begin{aligned}
\mathcal{C} &= \{x \in \mathbb{R}^n : h(x) \geq 0\} \\
\partial\mathcal{C} &= \{x \in \mathbb{R}^n : h(x) = 0\} \\
\text{Int}(\mathcal{C}) &= \{x \in \mathbb{R}^n : h(x) > 0\}
\label{eq:safe_set}
\end{aligned}
\end{equation}
ここで,$h: \mathbb{R}^n \rightarrow \mathbb{R}$は連続微分可能な関数である.

\begin{definition}[制御バリア関数]
連続微分可能な関数$h: \mathbb{R}^n \rightarrow \mathbb{R}$について,ある拡張クラス$\mathcal{K}_{\infty}$関数$\alpha$が存在し,任意の$x \in \mathcal{C}$に対して以下の条件を満たす制御入力$u \in \mathbb{R}^m$が存在するとき,$h$を制御バリア関数と呼ぶ.
\begin{equation}
\begin{aligned}
\sup_{u \in \mathbb{R}^m} [L_f h(x) + L_g h(x)u + \alpha(h(x))] \geq 0
\label{eq:cbf_condition}
\end{aligned}
\end{equation}
ここで,$L_f h(x) = \frac{\partial h}{\partial x}f(x)$および$L_g h(x) = \frac{\partial h}{\partial x}g(x)$はそれぞれLie微分を表す.
\end{definition}

CBFの条件\Eqref{eq:cbf_condition}を満たす制御入力$u$を用いると,システム\Eqref{eq:system}の解は常に安全集合$\mathcal{C}$内に留まることが保証される.つまり,$x(0) \in \mathcal{C}$ならば,任意の$t \geq 0$に対して$x(t) \in \mathcal{C}$となる.

実際の制御入力を求めるために,以下のような二次計画問題(QP)を解く.
\begin{equation}
\begin{aligned}
u^* = \arg\min_{u \in \mathbb{R}^m} & \|u - u_{\text{nom}}\|^2 \\
\text{s.t.} & L_f h(x) + L_g h(x)u + \alpha(h(x)) \geq 0
\label{eq:cbf_qp}
\end{aligned}
\end{equation}
ここで,$u_{\text{nom}}$は安全制約がない場合の公称制御入力である.この最適化問題は,安全制約を満たしつつ,公称制御入力からの偏差を最小化する制御入力を求めるものである.

\subsection{高次制御バリア関数}

従来のCBFは制約関数の相対次数が1であることを仮定していたが,多くの実システムでは安全制約が高次微分に依存するため,高次制御バリア関数(High Order Control Barrier Functions, HOCBF)が提案されている\cite{Xiao2022}.

関数$h(x)$の相対次数が$r > 1$の場合,以下のように補助関数の連鎖を定義する.
\begin{equation}
\begin{aligned}
\psi_0(x) &= h(x) \\
\psi_1(x) &= \dot{\psi}_0(x) + \alpha_1(\psi_0(x)) \\
\psi_2(x) &= \dot{\psi}_1(x) + \alpha_2(\psi_1(x)) \\
&\vdots \\
\psi_r(x) &= \dot{\psi}_{r-1}(x) + \alpha_r(\psi_{r-1}(x))
\label{eq:hocbf_chain}
\end{aligned}
\end{equation}
ここで,$\alpha_i$($i = 1, 2, \ldots, r$)は拡張クラス$\mathcal{K}$関数である.

\begin{definition}[高次制御バリア関数]
相対次数$r$の連続微分可能な関数$h: \mathbb{R}^n \rightarrow \mathbb{R}$について,\Eqref{eq:hocbf_chain}で定義される補助関数の連鎖に対して,任意の$x \in \mathcal{C}_r = \{x \in \mathbb{R}^n : \psi_i(x) \geq 0, i = 0, 1, \ldots, r-1\}$に対して以下の条件を満たす制御入力$u \in \mathbb{R}^m$が存在するとき,$h$を高次制御バリア関数と呼ぶ.
\begin{equation}
\begin{aligned}
\dot{\psi}_{r-1}(x) + \alpha_r(\psi_{r-1}(x)) \geq 0
\label{eq:hocbf_condition}
\end{aligned}
\end{equation}
\end{definition}

HOCBFを用いた制御入力を求めるための二次計画問題は以下のように定式化される.
\begin{equation}
\begin{aligned}
u^* = \arg\min_{u \in \mathbb{R}^m} & \|u - u_{\text{nom}}\|^2 \\
\text{s.t.} & L_f^r h(x) + L_g L_f^{r-1} h(x)u + O(h(x)) \geq 0
\label{eq:hocbf_qp}
\end{aligned}
\end{equation}
ここで,$L_f^r h(x)$は$h(x)$の$f(x)$に関する$r$次のLie微分,$L_g L_f^{r-1} h(x)$は$L_f^{r-1} h(x)$の$g(x)$に関するLie微分,$O(h(x))$は$h(x)$とその微分に依存する項である.

本研究では,SE(3)上の剛体運動に対して,視野共有を保証するためのCBFおよびHOCBFを設計し,分散型の実装を提案する.

\section{問題設定}

本章では,本研究で扱うシステムモデルと共有視野の定義,および問題設定について述べる.

\subsection{システムモデル:SE(3)上の剛体運動}

本研究では,複数のドローンがSE(3)上で運動する状況を考える.各エージェント$i \in \mathcal{A}$の位置と姿勢は,特殊ユークリッド群SE(3)上の要素$T_i = (p_i, R_i) \in \mathrm{SE}(3)$で表される.ここで,$p_i \in \mathbb{R}^3$は位置ベクトル,$R_i \in \mathrm{SO}(3)$は回転行列である.

SE(3)上の剛体運動は,以下の行列表現で記述できる:
\begin{equation}
\begin{aligned}
T_i &= \begin{bmatrix}
R_i & p_i \\
0 & 1
\end{bmatrix} \in \mathrm{SE}(3)
\label{eq:se3_matrix}
\end{aligned}
\end{equation}

ボディ座標系における速度入力$\xi^\wedge_{B,i}$によるSE(3)上の剛体運動式は以下のように表される:
\begin{equation}
\begin{aligned}
\dot{T}_i &= T_i \xi^\wedge_{B,i} \\
\xi^\wedge_{B,i} &= \begin{bmatrix}
[\omega_i]_\times & v_{B,i} \\
0 & 0
\end{bmatrix} \in \mathfrak{se}(3)
\label{eq:se3_dynamics_body}
\end{aligned}
\end{equation}
ここで,$\omega_i \in \mathbb{R}^3$は角速度ベクトル,$v_{B,i} \in \mathbb{R}^3$はボディ座標系における並進速度ベクトル,$[\cdot]_\times: \mathbb{R}^3 \rightarrow \mathfrak{so}(3)$は歪対称行列を生成する演算子である.

世界座標系における速度入力$\xi^\wedge_{W,i}$による剛体運動式は以下のように表される:
\begin{equation}
\begin{aligned}
\dot{T}_i &= \xi^\wedge_{W,i} T_i \\
\xi^\wedge_{W,i} &= \begin{bmatrix}
[R_i\omega_i]_\times & [p_i]_\times R_i\omega_i + R_iv_{B,i} \\
0 & 0
\end{bmatrix}
\label{eq:se3_dynamics_world}
\end{aligned}
\end{equation}

ボディ座標系と世界座標系の速度は,随伴写像$\mathrm{Ad}_{T_i}$を用いて以下のように変換できる:
\begin{equation}
\begin{aligned}
\xi_{W,i} &= \mathrm{Ad}_{T_i}\xi_{B,i} \\
&= \begin{bmatrix}
R_i & 0 \\
[p_i]_\times R_i & R_i
\end{bmatrix}
\begin{bmatrix}
\omega_i \\
v_{B,i}
\end{bmatrix} \\
&= \begin{bmatrix}
R_i\omega_i \\
[p_i]_\times R_i\omega_i + R_iv_{B,i}
\end{bmatrix}
\label{eq:adjoint_map}
\end{aligned}
\end{equation}

世界座標系における並進速度を$v_i = R_iv_{B,i}$とすると,\Eqref{eq:se3_dynamics_world}は以下のように簡略化できる:
\begin{equation}
\begin{aligned}
\dot{R}_i &= R_i[\omega_i]_\times \\
\dot{p}_i &= v_i
\label{eq:se3_dynamics_simplified}
\end{aligned}
\end{equation}

実際の制御設計では,離散時間システムとして扱うことが多い.ボディ座標系における運動方程式を離散化すると以下のようになる:
\begin{equation}
\begin{aligned}
T_{i,k+1} &\simeq T_{i,k} + \dot{T}_{i,k} \\
&\simeq T_{i,k} + h T_{i,k}\xi^\wedge_{B,i,k} \\
&= T_{i,k} + h\begin{bmatrix}
R_{i,k} & p_{i,k} \\
0 & 1
\end{bmatrix}
\begin{bmatrix}
[\omega_{i,k}]_\times & v_{B,i,k} \\
0 & 0
\end{bmatrix}
\label{eq:se3_discrete}
\end{aligned}
\end{equation}
ここで,$h$はサンプリング時間である.

成分分解すると,以下の離散時間更新式が得られる:
\begin{equation}
\begin{aligned}
R_{i,k+1} &\simeq R_{i,k}\exp(h[\omega_{i,k}]_{\times}) \\
&\simeq R_{i,k}(I + h[\omega_{i,k}]_{\times}) \\
p_{i,k+1} &= p_{i,k} + hR_{i,k}v_{B,i,k} \\
&= p_{i,k} + hv_{i,k}
\label{eq:se3_discrete_components}
\end{aligned}
\end{equation}
ここで,$v_{i,k} = R_{i,k}v_{B,i,k}$は世界座標系における並進速度である.

\subsection{共有視野の定義}

環境内の特徴点を$q_l \in \mathcal{L}$とし,エージェント$i$の観測している特徴点集合を$\mathcal{C}_i$とする.特徴点$q_l$がエージェント$i$の視野内にある条件は以下のように定義される:
\begin{equation}
\begin{aligned}
q_l \in \mathcal{C}_i \iff \beta_l^{\top}(p_i)R_ie_c - \cos\Psi_\mathcal{F} > 0
\label{eq:fov_condition}
\end{aligned}
\end{equation}
ここで,$\beta_l(p_i) = \frac{q_l-p_i}{\|q_l-p_i\|}$は特徴点$q_l$からエージェント$i$への単位方向ベクトル,
$e_c = [0\:0\:1]^\top$はカメラ方向を表す単位ベクトル,
$\Psi_{\mathcal{F}}$
は視野角(Field of View, FoV)である.

複数のエージェントが共通の特徴点を観測する条件(共有視野条件)は,以下のように定義される:
\begin{equation}
\begin{aligned}
q_l \in \mathcal{C}_i \cap \mathcal{C}_j \iff (\beta_l^{\top}(p_i)R_ie_c - \cos\Psi_\mathcal{F})(\beta_l^{\top}(p_j)R_je_c - \cos\Psi_\mathcal{F}) > 0
\label{eq:shared_fov_condition}
\end{aligned}
\end{equation}
この条件は,特徴点$q_l$が両方のエージェントの視野内にあることを意味する.

\subsection{問題設定}

本研究の目的は,複数のエージェントが常に共通の特徴点を観測できるよう,各エージェントの制御入力を設計することである.具体的には,以下の問題を考える:

\begin{dfn}[共有視野保証問題]
エージェント集合$\mathcal{A}$と特徴点集合$\mathcal{L}$が与えられたとき,各エージェント$i \in \mathcal{A}$に対して,以下の条件を満たす制御入力$\xi_{B,i} = (\omega_i, v_{B,i})$を設計せよ:
\begin{enumerate}
\item 各エージェント$i$は目標位置$p_i^d$に向かって移動する.
\item 各エージェント$i$と$j$の間で,少なくとも$m$個の共通特徴点が常に観測可能である.つまり,$|\mathcal{C}_i \cap \mathcal{C}_j| \geq m$が常に成り立つ.
\end{enumerate}
\end{dfn}

この問題に対して,本研究では制御バリア関数(CBF)を用いたアプローチを提案する.特に,特徴点の可視性を確率的に評価し,CBFに組み込むことで,不確実性下でも安全な視野共有を実現する.また,非ホロノミックなドローンダイナミクスに対応するため,高次制御バリア関数(HOCBF)を導入し,分散最適化アルゴリズムにより各エージェントが局所的に制御入力を計算する枠組みを提案する.

\section{CBFs for Shared Field of View}

本章では,共有視野を保証するための制御バリア関数(CBF)を設計する.まず単一の特徴点を追従する場合の安全制約を導出し,次に複数の特徴点を追従する場合,さらに複数のエージェントが共通の特徴点を追従する場合へと拡張する.最後に,分散型実装について述べる.

\subsection{単一特徴点追従の安全制約}

単一の特徴点を追従する安全制約については,Trimarchiら\cite{Trimarchi2025}の研究に大きく影響を受けている.

エージェント位置を$T_i=(p_i, R_i)\in \mathrm{SE}(3), i\in\mathcal{A}$,環境内の特徴点を$q_l\in \mathcal{L}$とする.エージェント$i$の観測している特徴点集合を$\mathcal{C}_i$とする.$q_l\in\mathcal{C}_i$である条件は,\Eqref{eq:fov_condition}で示したように以下のように表される.
\begin{equation}
\begin{aligned}
\beta_l^{\top}(p_i)R_ie_c-\cos\Psi_\mathcal{F}>0
\label{eq:fov_condition_single}
\end{aligned}
\end{equation}

この条件に基づいて,安全集合を以下のように定義する.
\begin{equation}
\begin{aligned}
B_{i} = \beta_l^{\top}(p_i)R_ie_c-\cos\Psi_\mathcal{F}
\label{eq:safe_set_single}
\end{aligned}
\end{equation}

安全集合$B_i$の時間微分は以下のように計算される.
\begin{equation}
\begin{aligned}
\dot{B}_{i} &= \langle \mathrm{grad}_R\:B_i,\omega_i\rangle + \langle \mathrm{grad}_p\:B_i,v_i\rangle \\
&= -\beta_l^\top(p_i) R_i [e_c]_\times\omega_i - \frac{e_c^\top R_i^\top P_{\beta_l}}{d_{i,l}}v_i
\label{eq:safe_set_derivative_single}
\end{aligned}
\end{equation}
ここで,$d_{i,l}={\|q_l-p_i\|}$は特徴点$q_l$とエージェント$i$の距離,$P_{\beta_l} = I-\beta_l\beta_l^\top$は$\beta_l$に直交する平面への投影行列である.また,$\mathrm{grad}_R\:B_i$と$\mathrm{grad}_p\:B_i$はそれぞれ$B_i$の$R_i$と$p_i$に関する勾配である.

CBFの条件より,安全制約は以下のように表される.
\begin{equation}
\begin{aligned}
-\beta_l^\top(p_i) R_i [e_c]_\times\omega_i - \frac{e_c^\top R_i^\top P_{\beta_l}}{d_{i,l}}v_i \leq \gamma (\beta_l^{\top}(p_i)R_ie_c-\cos\Psi_\mathcal{F})
\label{eq:cbf_constraint_single}
\end{aligned}
\end{equation}
ここで,$\gamma > 0$はCBFのゲインパラメータである.

CBF制約を満たしつつ,目標位置$p_i^d$に追従するためのQPは以下のように定式化される.
\begin{equation}
\begin{aligned}
\min_{\xi_i} \: (p^d_{i}-p_{i,{k}}-hR_{i,k}v_{i,k})^\top Q_1 & (p^d_{i}-p_{i,{k}}-hR_{i,k}v_{i,k})
+ 
\begin{bmatrix}
\omega_{i,k}\\v_{i,k}
\end{bmatrix}^\top Q_2
\begin{bmatrix}
\omega_{i,k}\\v_{i,k}
\end{bmatrix} \\
\mathrm{s.t.} \quad
\beta_l^\top(p_i) R_i [e_c]_\times\omega_i &+ \frac{e_c^\top R_i^\top P_{\beta_l}}{d_{i,l}}v_i \leq \gamma (\beta_l^{\top}(p_i)R_ie_c-\cos\Psi_\mathcal{F})
\label{eq:qp_single}
\end{aligned}
\end{equation}
ここで,$Q_1$と$Q_2$は正定値重み行列である.

この最適化問題を一般的な形式に変換すると,以下のようになる.
\begin{equation}
\begin{aligned}
&\min_{\xi_i}\:J_i \\
J_i &= \frac{1}{2}\xi_{i,k}^\top H_i \xi_{i,k} + f_i^\top \xi_{i,k} \\
\xi_{i,k} &= \begin{bmatrix}
\omega_{i,k}\\v_{i,k}
\end{bmatrix}, \quad
H_i = 2\begin{bmatrix}
Q_{2,\omega} & Q_{2,\omega v} \\ 
Q^\top_{2,\omega v} & Q_{2,v}+h^2R_{i,k}^\top Q_1R_{i,k}
\end{bmatrix}, \\
f_i &= \begin{bmatrix}
0 \\ -2hR_i^\top Q_1 e_i
\end{bmatrix}, \quad e_i = p^d_{i}-p_{i,k} \\
\mathrm{s.t.} &\quad
\begin{bmatrix}
\beta_l^\top(p_i) R_i [e_c]_\times \\
\frac{e_c^\top R_i^\top P_{\beta_l}}{d_{i,l}}
\end{bmatrix}^\top \xi_{i,k} \leq
\gamma (\beta_l^{\top}(p_i)R_ie_c-\cos\Psi_\mathcal{F})
\label{eq:qp_single_general}
\end{aligned}
\end{equation}

\subsection{複数特徴点追従の安全制約}

複数の特徴点を追従することを考える場合,単純に各特徴点に対する安全集合の積集合を考えることもできるが,それでは安全集合が微分不可能になる可能性がある.そこで,確率的なアプローチを導入する.

特徴点$q_l \in \mathcal{L}$によってエージェント$i$における推定が成り立っている確率を以下のように定義する.
\begin{equation}
\begin{aligned}
\phi_{i}^l = P(p_i,R_i,q_l)
\label{eq:probability_single}
\end{aligned}
\end{equation}

この確率を用いて,新しい安全集合を以下のように定義する.
\begin{equation}
\begin{aligned}
B_{i} = 1-q-\prod_{l\in\mathcal{L}}(1-\phi_{i}^l)
\label{eq:safe_set_multi}
\end{aligned}
\end{equation}
ここで,$q \in (0,1)$はパラメータであり,環境内の特徴点$q_l \in \mathcal{L}$によってエージェント$i$における推定が達成される確率を$q$以上に制限することを意味する.

確率$\phi_{i}^l$は以下のように定義する.
\begin{equation}
\begin{aligned}
P(p_i,R_i,q_l) &= \left\{ \begin{array}{ll}
P_i^l & \mathrm{if} \quad q_l\in\mathcal{C}_i \\
0 & \mathrm{if} \quad q_l\in \mathcal{L}\setminus\mathcal{C}_i
\end{array} \right. \\
\mathrm{where} \quad P_i^l &= \frac{\beta_l^\top(p_i) R_i e_c -\cos\Psi_\mathcal{F}}{1-\cos\Psi_\mathcal{F}}
\label{eq:probability_definition}
\end{aligned}
\end{equation}

安全制約は以下のように表される.
\begin{equation}
\begin{aligned}
B_{i} > 0 \quad \forall i \in \mathcal{A}
\label{eq:safe_constraint_multi}
\end{aligned}
\end{equation}

安全集合$B_i$の時間微分は以下のように計算される.
\begin{equation}
\begin{aligned}
B_{i} &= 1-q-\eta_{i} \\
\eta_{i} &= \prod_{l\in\mathcal{L}}(1-\phi_{i}^l) \\
\dot{B}_{i} &= -\dot{\eta}_{i} \\
&= -\frac{d}{dt}\prod_{l\in\mathcal{L}}(1-\phi_{i}^l) \\
&= \sum_{l\in \mathcal{L}}(\prod_{k\neq l}(1-\phi_{i}^k))\dot{\phi}^l_{i} \\
\dot{\phi}^l_{i} &= \left\{ \begin{array}{ll}
\dot{P}_i^l & \mathrm{if} \quad q_l\in\mathcal{C}_i \\
0 & \mathrm{if} \quad q_l\in \mathcal{L}\setminus\mathcal{C}_i
\end{array} \right.
\label{eq:safe_set_derivative_multi}
\end{aligned}
\end{equation}

$P_i^l$の微分は以下のように計算できる.
\begin{equation}
\begin{aligned}
P_i^l &= \frac{\beta_l^\top(p_i) R_i e_c -\cos\Psi_\mathcal{F}}{1-\cos\Psi_\mathcal{F}} \\
\dot{P}_i^l &= \langle \mathrm{grad}\:P_i^l, \xi_W\rangle \\
&= \left\langle 
\begin{bmatrix}\mathrm{grad}_R\:P_i^l\\\mathrm{grad}_p\:P_i^l
\end{bmatrix},
\begin{bmatrix}\omega_i\\v_i
\end{bmatrix}
\right\rangle \\
\mathrm{grad}_p\:P_i^l &= \frac{1}{1-\cos \Psi_\mathcal{F}}\left(-\frac{e_c^\top R_i^\top P_{\beta_l}}{d_{i,l}}\right) \\
\mathrm{grad}_R\:P_i^l &= \frac{1}{1-\cos \Psi_\mathcal{F}}(-\beta_l^\top(p_i) R_i [e_c]_\times)
\label{eq:probability_derivative}
\end{aligned}
\end{equation}

これらから,複数の特徴点を追従するためのCBF制約は以下のようになる.
\begin{equation}
\begin{aligned}
\sum_{l\in \mathcal{L}\cap\mathcal{C}_i}(\prod_{k\neq l}(1-\phi_{i}^k)) \langle \mathrm{grad}_R\:P_i^l,\omega_i\rangle &+ \sum_{l\in \mathcal{L}\cap\mathcal{C}_i}(\prod_{k\neq l}(1-\phi_{i}^k)) \langle \mathrm{grad}_p\:P_i^l,v_i\rangle \\
&\geq -\gamma_0 (1-q-\prod_{l\in\mathcal{L}}(1-\phi_{i}^l))
\label{eq:cbf_constraint_multi}
\end{aligned}
\end{equation}
ここで,$\gamma_0 > 0$はCBFのゲインパラメータである.

CBF制約を満たしつつ,目標位置$p_i^d$に追従するためのQPは以下のように定式化される.
\begin{equation}
\begin{aligned}
\min_{\xi_i} \: (p^d_{i}-p_{i,{k+1}})^\top Q_1 & (p^d_{i}-p_{i,{k+1}})
+ \xi_{i,k}^\top Q_2 \xi_{i,k} \\
\mathrm{s.t.} \quad
\sum_{l\in \mathcal{L}\cap\mathcal{C}_i}(\prod_{k\neq l}(1-\phi_{i}^k)) \frac{\beta_l^\top(p_i) R_i [e_c]_\times}{1-\cos \Psi_\mathcal{F}}\omega_i &+ \sum_{l\in \mathcal{L}\cap\mathcal{C}_i}(\prod_{k\neq l}(1-\phi_{i}^k)) \frac{e_c^\top R_i^\top P_{\beta_l}}{(1-\cos \Psi_\mathcal{F})d_{i,l}}v_i \\
&\leq \gamma_0 (1-q-\prod_{l\in\mathcal{L}}(1-\phi_{i}^k))
\label{eq:qp_multi}
\end{aligned}
\end{equation}

一般的なQPの形式に変換すると,以下のようになる.
\begin{equation}
\begin{aligned}
&\min_{\xi_i}\:J_i \\
J_i &= \frac{1}{2}\xi_{i,k}^\top H_i \xi_{i,k} + f_i^\top \xi_{i,k} \\
\xi_{i,k} &= \begin{bmatrix}
\omega_{i,k}\\v_{i,k}
\end{bmatrix}, \quad
H_i = 2\begin{bmatrix}
Q_{2,\omega} & Q_{2,\omega v} \\ 
Q^\top_{2,\omega v} & Q_{2,v}+h^2R_{i,k}^\top Q_1R_{i,k}
\end{bmatrix}, \\
f_i &= \begin{bmatrix}
0 \\ -2hR_i^\top Q_1 e_i
\end{bmatrix}, \quad e_i = p^d_{i}-p_{i,k} \\
\mathrm{s.t.} &\quad
\begin{bmatrix}
\sum_{l\in \mathcal{L}\cap\mathcal{C}_i}(\prod_{k\neq l}(1-\phi_{i}^k)) \frac{\beta_l^\top(p_i) R_i [e_c]_\times}{1-\cos \Psi_\mathcal{F}} \\
\sum_{l\in \mathcal{L}\cap\mathcal{C}_i}(\prod_{k\neq l}(1-\phi_{i}^k))\frac{e_c^\top R_i^\top P_{\beta_l}}{(1-\cos \Psi_\mathcal{F})d_{i,l}}
\end{bmatrix}^\top \xi_{i,k} \leq
\gamma_0(1-q-\prod_{l\in \mathcal{L}}(1-\phi_{i}^k))
\label{eq:qp_multi_general}
\end{aligned}
\end{equation}

\subsection{複数エージェントの共通特徴点追従}

次に,複数のエージェントが共通の特徴点を追従する場合を考える.エージェント$i$と$j$が特徴点$q_l$を共有視野内に持つ条件は,\Eqref{eq:shared_fov_condition}で示したように以下のように表される.
\begin{equation}
\begin{aligned}
(\beta_l^{\top}(p_i)R_ie_c-\cos\Psi_\mathcal{F})(\beta_l^{\top}(p_j)R_je_c-\cos\Psi_\mathcal{F}) > 0
\label{eq:shared_fov_condition_repeat}
\end{aligned}
\end{equation}

特徴点$q_l \in \mathcal{L}$によってエッジ$(i,j) \in \mathcal{E}$における推定が成り立っている確率を以下のように定義する.
\begin{equation}
\begin{aligned}
\phi_{ij}^l = P(p_i,R_i,p_j,R_j,q_l)
\label{eq:probability_edge}
\end{aligned}
\end{equation}

この確率を用いて,新しい安全集合を以下のように定義する.
\begin{equation}
\begin{aligned}
B_{ij} = 1-q-\prod_{l\in\mathcal{L}}(1-\phi_{ij}^l)
\label{eq:safe_set_edge}
\end{aligned}
\end{equation}
ここで,$q \in (0,1)$はパラメータであり,環境内の特徴点$q_l \in \mathcal{L}$によってエッジ$(i,j) \in \mathcal{E}$における推定が達成される確率を$q$以上に制限することを意味する.

確率$\phi_{ij}^l$は以下のように定義する.
\begin{equation}
\begin{aligned}
P(p_i,R_i,p_j,R_j,q_l) &= \left\{ \begin{array}{ll}
P_i^l P_j^l & \mathrm{if} \quad q_l\in\mathcal{C}_i \cap \mathcal{C}_j \\
0 & \mathrm{if} \quad q_l\in \mathcal{L}\setminus(\mathcal{C}_i \cap \mathcal{C}_j)
\end{array} \right. \\
\mathrm{where} \quad P_i^l &= \frac{\beta_l^\top(p_i) R_i e_c -\cos\Psi_\mathcal{F}}{1-\cos\Psi_\mathcal{F}}
\label{eq:probability_edge_definition}
\end{aligned}
\end{equation}

安全制約は以下のように表される.
\begin{equation}
\begin{aligned}
B_{ij} > 0 \quad \forall (i,j) \in \mathcal{E}
\label{eq:safe_constraint_edge}
\end{aligned}
\end{equation}

安全集合$B_{ij}$の時間微分は以下のように計算される.
\begin{equation}
\begin{aligned}
B_{ij} &= 1-q-\eta_{ij} \\
\eta_{ij} &= \prod_{l\in\mathcal{L}}(1-\phi_{ij}^l) \\
\dot{B}_{ij} &= -\dot{\eta}_{ij} \\
&= -\frac{d}{dt}\prod_{l\in\mathcal{L}}(1-\phi_{ij}^l) \\
&= \sum_{l\in \mathcal{L}}(\prod_{k\neq l}(1-\phi_{ij}^k))\dot{\phi}^l_{ij} \\
\dot{\phi}^l_{ij} &= \left\{ \begin{array}{ll}
\dot{P}_i^l P_j^l + P_i^l \dot{P}_j^l & \mathrm{if} \quad q_l\in\mathcal{C}_i \cap \mathcal{C}_j \\
0 & \mathrm{if} \quad q_l\in \mathcal{L}\setminus(\mathcal{C}_i \cap \mathcal{C}_j)
\end{array} \right.
\label{eq:safe_set_derivative_edge}
\end{aligned}
\end{equation}

エージェントごとの制御入力について分解すると,以下のようになる.
\begin{equation}
\begin{aligned}
\dot{B}_{ij} &= -\dot{\eta}_{ij} \\
&= -\frac{d}{dt}\prod_{l\in\mathcal{L}}(1-\phi_{ij}^l) \\
&= \sum_{l\in \mathcal{L}}(\prod_{k\neq l}(1-\phi_{ij}^k))\dot{\phi}^l_{ij} \\
&= \sum_{l\in \mathcal{L}\cap\mathcal{C}_i \cap \mathcal{C}_j}(\prod_{k\neq l}(1-\phi_{ij}^k))P_j^l\dot{P}_i^l + \sum_{l\in \mathcal{L}\cap\mathcal{C}_i \cap \mathcal{C}_j}(\prod_{k\neq l}(1-\phi_{ij}^k))P_i^l\dot{P}_j^l
\label{eq:safe_set_derivative_edge_decomposed}
\end{aligned}
\end{equation}

これらから,複数のエージェントが共通の特徴点を追従するためのCBF制約は以下のようになる.
\begin{equation}
\begin{aligned}
&\sum_{l\in \mathcal{L}\cap\mathcal{C}_i \cap \mathcal{C}_j}(\prod_{k\neq l}(1-\phi_{ij}^k))P_j^l \langle \mathrm{grad}_R\:P_i^l,\omega_i\rangle + \sum_{l\in \mathcal{L}\cap\mathcal{C}_i \cap \mathcal{C}_j}(\prod_{k\neq l}(1-\phi_{ij}^k))P_i^l \langle \mathrm{grad}_R\:P_j^l,\omega_j \rangle \\
&+ \sum_{l\in \mathcal{L}\cap\mathcal{C}_i \cap \mathcal{C}_j}(\prod_{k\neq l}(1-\phi_{ij}^k))P_j^l \langle \mathrm{grad}_p\:P_i^l,v_i\rangle + \sum_{l\in \mathcal{L}\cap\mathcal{C}_i \cap \mathcal{C}_j}(\prod_{k\neq l}(1-\phi_{ij}^k))P_i^l \langle \mathrm{grad}_p\:P_j^l, v_j\rangle \\
&\geq -\gamma_0 (1-q-\prod_{l\in\mathcal{L}}(1-\phi_{ij}^l))
\label{eq:cbf_constraint_edge}
\end{aligned}
\end{equation}
ここで,$\gamma_0 > 0$はCBFのゲインパラメータである.

CBF制約を満たしつつ,目標位置$p_i^d$と$p_j^d$に追従するためのQPは以下のように定式化される.
\begin{equation}
\begin{aligned}
\min_{\xi_i, \xi_j} \: \sum_{i,j}(p^d_{i}-p_{i,{k+1}}-hR_{i,k}v_{i,k})^\top Q_1 & (p^d_{i}-p_{i,{k+1}}-hR_{i,k}v_{i,k})
+ 
\begin{bmatrix}
\omega_{i,k}\\v_{i,k}
\end{bmatrix}^\top Q_2
\begin{bmatrix}
\omega_{i,k}\\v_{i,k}
\end{bmatrix} \\
\mathrm{s.t.} \quad
&\sum_{l\in \mathcal{L}\cap\mathcal{C}_i \cap \mathcal{C}_j}(\prod_{k\neq l}(1-\phi_{ij}^k)) \frac{P_j^l\beta_l^\top(p_i) R_i [e_c]_\times}{1-\cos \Psi_\mathcal{F}}\omega_i \\
&+ \sum_{l\in \mathcal{L}\cap\mathcal{C}_i \cap \mathcal{C}_j}(\prod_{k\neq l}(1-\phi_{ij}^k)) 
\frac{P_i^l\beta_l^\top(p_j) R_j [e_c]_\times}{1-\cos \Psi_\mathcal{F}}\omega_j \\
&+ \sum_{l\in \mathcal{L}\cap\mathcal{C}_i \cap \mathcal{C}_j}(\prod_{k\neq l}(1-\phi_{ij}^k))\frac{P_j^le_c^\top R_i^\top P_{\beta_l}}{(1-\cos \Psi_\mathcal{F})d_{i,l}}v_i \\
&+ \sum_{l\in \mathcal{L}\cap\mathcal{C}_i \cap \mathcal{C}_j}(\prod_{k\neq l}(1-\phi_{ij}^k))\frac{P_i^le_c^\top R_j^\top P_{\beta_l}}{(1-\cos \Psi_\mathcal{F})d_{j,l}}v_j \\
&\leq \gamma_0 (1-q-\prod_{l\in\mathcal{L}}(1-\phi_{ij}^l))
\label{eq:qp_edge}
\end{aligned}
\end{equation}

一般的なQPの形式に変換すると,以下のようになる.
\begin{equation}
\begin{aligned}
&\min_{\xi_i, \xi_j} \: \sum_{i,j}J_i \\
J_i &= \frac{1}{2}\xi_{i,k}^\top H_i\xi_{i,k} + f_i^\top\xi_{i,k} \\
\xi_{i,k} &= \begin{bmatrix}
\omega_{i,k}\\v_{i,k}
\end{bmatrix}, \quad
H_i = 2\begin{bmatrix}
Q_{2,\omega} & Q_{2,\omega v} \\ 
Q^\top_{2,\omega v} & Q_{2,v}+h^2R_{i,k}^\top Q_1R_{i,k}
\end{bmatrix}, \\
f_i &= \begin{bmatrix}
0 \\ -2hR_i^\top Q_1 e_i
\end{bmatrix}, \quad e_i = p^d_{i}-p_{i,k} \\
\mathrm{s.t.} &\quad \begin{bmatrix}
\alpha_\omega & \alpha_v & \beta_\omega & \beta_v
\end{bmatrix}
\begin{bmatrix}
\omega_{i,k} \\ v_{i,k} \\ \omega_{j,k} \\ v_{j,k}
\end{bmatrix} \leq
\gamma_0\gamma
\label{eq:qp_edge_general}
\end{aligned}
\end{equation}
ここで,係数は以下のように定義される.
\begin{equation}
\begin{aligned}
\alpha_\omega &= \sum_{l\in \mathcal{L}\cap\mathcal{C}_i \cap \mathcal{C}_j}(\prod_{k\neq l}(1-\phi_{ij}^k)) \frac{P_j^l\beta_l^\top(p_i) R_i [e_c]_\times}{1-\cos \Psi_\mathcal{F}} \in \mathbb{R}^3 \\
\beta_\omega &= \sum_{l\in \mathcal{L}\cap\mathcal{C}_i \cap \mathcal{C}_j}(\prod_{k\neq l}(1-\phi_{ij}^k)) 
\frac{P_i^l\beta_l^\top(p_j) R_j [e_c]_\times}{1-\cos \Psi_\mathcal{F}} \in \mathbb{R}^3 \\
\alpha_v &= \sum_{l\in \mathcal{L}\cap\mathcal{C}_i \cap \mathcal{C}_j}(\prod_{k\neq l}(1-\phi_{ij}^k))\frac{P_j^le_c^\top R_i^\top P_{\beta_l}}{(1-\cos \Psi_\mathcal{F})d_{i,l}} \in \mathbb{R}^3 \\
\beta_v &= \sum_{l\in \mathcal{L}\cap\mathcal{C}_i \cap \mathcal{C}_j}(\prod_{k\neq l}(1-\phi_{ij}^k))\frac{P_i^le_c^\top R_j^\top P_{\beta_l}}{(1-\cos \Psi_\mathcal{F})d_{j,l}} \in \mathbb{R}^3 \\
\gamma &= 1-q-\prod_{l\in \mathcal{L}\cap\mathcal{C}_i \cap \mathcal{C}_j}(1-\phi_{ij}^l) \in \mathbb{R}
\label{eq:qp_edge_coefficients}
\end{aligned}
\end{equation}

\subsection{分散型実装}

不等式制約付き最適化問題を分散化する方法はいくつかあるが,本研究では主双対乗数法(Primal-Dual Method of Multipliers, PDMM)を用いる.PDMMは,不等式制約を処理する場合にADMMのようにスラック変数修正が必要なく,より効率的に解くことができる\cite{Zhang2024}.

IEQ-PDMM(inequality constraint primal-dual method of multipliers)を用いて上の式を分散化すると以下のようになる.
\begin{equation}
\begin{aligned}
\xi_i &= \underset{\xi_i}{\mathrm{argmin}} \:J_i(\xi_i)+z_{i|j}^\top A_{ij}\xi_i+\frac{c}{2}\|A_{ij}\xi_i-\frac{1}{2}\gamma_0\gamma\|^2 \\
y_{i|j} &= z_{i|j}+2c(A_{ij}\xi_{i}-\frac{1}{2}\gamma_0\gamma) \\
&\mathbf{node}_j\leftarrow \mathbf{node}_i(y_{i|j}) \\
&\mathbf{if}\:y_{i|j}+y_{j|i}>0 \\
&\qquad z_{i|j}=y_{j|i} \\
&\mathbf{else}\: \\
&\qquad z_{i|j}=-y_{i|j}
\label{eq:pdmm}
\end{aligned}
\end{equation}
ここで,$A_{ij} = \begin{bmatrix} \alpha_\omega \\ \alpha_v \end{bmatrix}$,$A_{ji} = \begin{bmatrix} \beta_\omega \\ \beta_v \end{bmatrix}$である.

QPの形式に表すと,以下のようになる.
\begin{equation}
\begin{aligned}
&\min_{\xi_i}\:\hat{J}_i = \frac{1}{2}\xi_{i,k}^\top \hat{H}_i\xi_{i,k} + \hat{f}_i^\top \xi_{i,k} \\
\xi_{i,k} &= \begin{bmatrix}
\omega_{i,k}\\v_{i,k}
\end{bmatrix}, \quad
\hat{H}_i = \begin{bmatrix}
2Q_{2,\omega}+c\alpha_\omega^\top \alpha_\omega & 2Q_{2,\omega v}+c\alpha_\omega^\top \alpha_v \\
2Q^\top_{2,\omega v}+c\alpha_v^\top \alpha_\omega & 2Q_{2,v}+2h^2R_{i,k}^\top Q_1R_{i,k}+c\alpha_v^\top \alpha_v
\end{bmatrix}, \\
\hat{f}_i &= \begin{bmatrix}
z_{i|j}\alpha_\omega^\top -\frac{c}{2}\gamma_0\gamma\alpha_\omega^\top \\
-2hR_i^\top Q_1 e_i+z_{i|j}\alpha_v^\top -\frac{c}{2}\gamma_0\gamma\alpha_v^\top
\end{bmatrix}, \quad e_i = p^d_{i}-p_{i,k}
\label{eq:pdmm_qp}
\end{aligned}
\end{equation}

この計算手法では制約式がすべてソフト制約化するため,最適化中に元の制約式を満たすことを保証できない.そこで,Tanら\cite{Tan2022}が提案したCBF-induced QPを用いることも考えられる.これは,ADMMベースのCBF分散化手法であり,最適化中に元の制約式を満たすことを保証する.

本章では,共有視野を保証するためのCBFを設計し,それに基づく制御則を提案した.次章では,二次系システムのための高次CBFを導入し,より実際的なドローンダイナミクスに対応する手法を提案する.

\section{Higher-Order CBFs for Second-Order Systems}

共有視野の分散CBFを実機に適用するため,本章ではドローンのダイナミクスを考慮したCBF付きQPの定式化を行う.まず,SE(3)における離散ダイナミクスを導出し,次にホロノミック系とQP定式化,非ホロノミック系への拡張,そして複数エージェント・複数特徴点の場合について述べる.

\subsection{SE(3)における離散ダイナミクス}

SE(3)における離散ダイナミクスは以下のように表される.
\begin{equation}
\begin{aligned}
R_{k+1} &= R_k F_k \\
p_{k+1} &= p_k + h R_k v_k \\
M v_{k+1} &= F_k^\top M v_k + h \mathcal{U}_{k+1} + h f_k \\
J \Omega_{k+1} &= F_k^\top J \Omega_{k} + h M v_{k+1} \times v_{k+1} + h \mathcal{M}_{k+1} + h \tau_k
\label{eq:se3_discrete_dynamics}
\end{aligned}
\end{equation}
ここで,$F_k$は回転行列の更新を表す行列,$M$は質量行列,$J$は慣性モーメント行列,$\mathcal{U}$と$\mathcal{M}$はそれぞれ位置と姿勢に関するポテンシャル力,$f_k$と$\tau_k$はそれぞれ並進力とトルクである.

ポテンシャル力は以下のように定義される.
\begin{equation}
\begin{aligned}
\mathcal{U}(p,R) &= -R^\top \frac{\partial U}{\partial p}(p,R) \\
\mathcal{M}(p,R)^\times &= \frac{\partial U}{\partial R}^\top R - R^\top \frac{\partial U}{\partial R} \\
U &= m g p_z
\label{eq:potential_forces}
\end{aligned}
\end{equation}
ここで,$U$はポテンシャルエネルギー,$m$は質量,$g$は重力加速度,$p_z$は高度である.

また,一般的な近似として以下を用いる.
\begin{equation}
\begin{aligned}
F_k \simeq \exp(h \Omega_k^\times) \simeq I + h \Omega_k^\times
\label{eq:rotation_approximation}
\end{aligned}
\end{equation}

$f, \tau$を制御入力とし,$p_{k+1}$を$u_k = (f_k, \tau_k)$についての線形な関数として表すと,並進運動について以下のようになる.
\begin{equation}
\begin{aligned}
p_{k+1} &= p_k + h R_k v_k \\
M v_{k+1} &= F_k^\top M v_k + h \mathcal{U}_{k+1} + h f_k \\
&= M v_k + h (M v_k \times \Omega_k - M g R^\top e_z + f)
\label{eq:translation_dynamics}
\end{aligned}
\end{equation}

回転運動について,後述する非ホロノミック系に対応するため以下のような近似を行う.
\begin{equation}
\begin{aligned}
R_{k+1} &= R_k F_k \simeq R_k F_{k+1} \\
p_{k+1} &= p_k + h R_{k} v_{k} \simeq p_k + h R_{k+1} v_{k+1}
\label{eq:rotation_approximation_2}
\end{aligned}
\end{equation}

この仮定のもとで,姿勢更新について以下のようになる.
\begin{equation}
\begin{aligned}
J \Omega_{k+1} &= F_k^\top J \Omega_{k} + \underbrace{h M v_{k+1} \times v_{k+1}}_{\simeq 0} + \underbrace{h \mathcal{M}_{k+1}}_{= 0} + h \tau_k \\
&= J \Omega_k + \underbrace{h J \Omega_k \times \Omega_k}_{\simeq 0} + h \tau_k \\
R_{k+1} &= R_{k} + h R_{k} \Omega_{k+1}^\times \\
&= R_{k} + h R_{k} [\Omega_k + h J^{-1} \tau_k]_\times \\
&= R_k + h R_k \Omega_k^\times + h^2 R_{k} [J^{-1} \tau_k]_\times
\label{eq:rotation_update}
\end{aligned}
\end{equation}

位置更新について以下のようになる.
\begin{equation}
\begin{aligned}
p_{k+1} &= p_k + h R_{k+1} v_{k+1} \\
&= p_k + h (R_k + h R_k \Omega_{k+1}^\times) (v_k + h (v_k \times \Omega_{k} - g R_k^\top e_z + R_k M^{-1} f)) \\
&= p_k + h (R_k + h R_k \Omega_k^\times + h^2 [J^{-1} \tau_k]_\times) (v_k + h (v_k \times \Omega_{k} - g R_k^\top e_z + R_k M^{-1} f)) \\
&= p_k + h R_k v_k + h^2 (-g e_z + R_k M^{-1} f) + h^3 [J^{-1} \tau_k]_\times v_k + \mathcal{O}(h^4)
\label{eq:position_update}
\end{aligned}
\end{equation}
これにより,逐次ステップでトルク$\tau_k$によって状態$(p_{k+1}, R_{k+1}) \in \mathrm{SE}(3)$を操作可能になる.

\subsection{ホロノミック系とQP定式化}

まず,$f = (f_x, f_y, f_z) \in \mathbb{R}^3$(ホロノミック系)におけるHOCBF-QPを考える.

目標位置$p^d_k$と現在位置$p_{k+1}$の誤差を以下のように定義する.
\begin{equation}
\begin{aligned}
p^d_{k} - p_{k+1} &= \underbrace{p^d_{k} - p_k - h R_k v_k}_{e_k} - h^2 (-g e_z + R_k M^{-1} f) - h^3 [J^{-1} \tau_k]_\times v_k \\
&= \underbrace{e_k + h^2 g e_z}_{\tilde{e}_k} - h^2 M^{-1} R_k f + h^3 v_k^\times J^{-1} \tau_k \\
&= \tilde{e}_k - h^2 M^{-1} R_k f + h^3 v_k^\times J^{-1} \tau_k \\
&= \tilde{e}_k + A_f f_k + A_\tau \tau_k \\
&= \tilde{e}_k + \begin{bmatrix} A_f & A_\tau \end{bmatrix} \begin{bmatrix} f_k \\ \tau_k \end{bmatrix} \\
&= \tilde{e}_k + A u_k
\label{eq:position_error}
\end{aligned}
\end{equation}
ここで,$A_f = -h^2 M^{-1} R_k$,$A_\tau = h^3 v_k^\times J^{-1}$,$u_k = [f_k^\top, \tau_k^\top]^\top$である.

最小化したい目的関数は以下のようになる.
\begin{equation}
\begin{aligned}
J &= \frac{1}{2} \|\tilde{e}_k + A u_k\|^2 + \frac{1}{2} \begin{bmatrix} A_g + M^{-1} f \\ J^{-1} \tau \end{bmatrix}^\top B \begin{bmatrix} A_g + M^{-1} f \\ J^{-1} \tau \end{bmatrix} \\
&\propto \frac{1}{2} u_k^\top A^\top A u_k + (A^\top \tilde{e}_k)^\top u_k \\
&\quad + \frac{1}{2} u_k^\top \underbrace{\begin{bmatrix} M^{-2} B_1 & 0 \\ 0 & (J^{-1})^\top B_2 J^{-1} \end{bmatrix}}_{B'_1 \in \mathbb{R}^{6 \times 6}} u_k + \underbrace{\begin{bmatrix} M^{-1} B_1 A_g^\top e_z \\ 0 \end{bmatrix}^\top}_{B'_2 \in \mathbb{R}^{6}} u_k \\
&\propto \frac{1}{2} u_k^\top (A^\top A + B'_1) u_k + (A^\top \tilde{e}_k + B'_2)^\top u_k
\label{eq:objective_function}
\end{aligned}
\end{equation}
ここで,$A_g = v^\times \Omega - g R^\top e_z$である.

よって解くべきQPは以下のようになる.
\begin{equation}
\begin{aligned}
\min_{u_k} &\: \frac{1}{2} u_k^\top (A^\top A + B'_1) u_k + (A^\top \tilde{e}_k + B'_2)^\top u_k \\
u_k &= \begin{bmatrix} f_k \\ \tau_k \end{bmatrix} \in \mathbb{R}^6 \\
A &= \begin{bmatrix} -h^2 M^{-1} R_k & h^3 v_k \times J^{-1} \end{bmatrix} \in \mathbb{R}^{3 \times 6} \\
B'_1 &= \begin{bmatrix} M^{-2} B_1 & 0 \\ 0 & (J^{-1})^\top B_2 J^{-1} \end{bmatrix} \in \mathbb{R}^{6 \times 6} \\
B'_2 &= \begin{bmatrix} M^{-1} B_1 A_g^\top e_z \\ 0 \end{bmatrix} \in \mathbb{R}^{6} \\
\tilde{e}_k &= p^d_{k} - p_k - h R_k v_k + h^2 g e_z
\label{eq:qp_holonomic}
\end{aligned}
\end{equation}

\subsection{単一の特徴点を追従するHOCBF制約}

前章で導入した安全集合$h = \beta_l^{\top}(p_i) R_i e_c - \cos \Psi_\mathcal{F}$に対して,二階微分を計算する.
\begin{equation}
\begin{aligned}
\ddot{h} &= \underbrace{-\frac{e_c^\top R^\top P_\beta R}{d} \dot{v}}_{\langle \mathrm{grad}_p h, \dot{v} \rangle} + \underbrace{\frac{d}{dt} \left( -\frac{e_c^\top R^\top P_\beta R}{d} \right) v}_{\langle \mathrm{Hess}_p h[v], v \rangle + \langle \mathrm{Hess}_p h[v], \Omega \rangle} \\
&\quad + \underbrace{\left( \frac{P_\beta}{d} R v \right)^\top R e_c^\times \Omega}_{\langle \mathrm{Hess}_R h[\Omega], v \rangle} + \underbrace{-\beta^\top R \Omega^\times e_c^\times \Omega}_{\langle \mathrm{Hess}_R h[\Omega], \Omega \rangle} + \underbrace{-\beta^\top R e_c^\times \dot{\Omega}}_{\langle \mathrm{grad}_R h, \dot{\Omega} \rangle} \\
&= \langle \mathrm{grad}_p h, \dot{v} \rangle + \langle \mathrm{Hess}_p h[v], v \rangle + \langle \mathrm{Hess}_p h[v], \Omega \rangle \\
&\quad + \langle \mathrm{grad}_R h, \dot{\Omega} \rangle + \langle \mathrm{Hess}_R h[\Omega], v \rangle + \langle \mathrm{Hess}_R h[\Omega], \Omega \rangle
\label{eq:h_second_derivative}
\end{aligned}
\end{equation}
ここで,各項は以下のように計算される.
\begin{equation}
\begin{aligned}
\langle \mathrm{Hess}_p h[v], \Omega \rangle &= \langle \mathrm{Hess}_R h[\Omega], v \rangle = v^\top R^\top \frac{P_\beta R e_c^\times}{d} \Omega \\
\langle \mathrm{Hess}_R h[\Omega], \Omega \rangle &= \Omega^\top [R^\top \beta]_\times e_c^\times \Omega \\
\mathrm{grad}_p h &= -\frac{e_c^\top R^\top P_\beta R}{d} \\
\mathrm{grad}_R h &= -\beta^\top R e_c^\times
\label{eq:h_derivatives}
\end{aligned}
\end{equation}

また,$\frac{d}{dt} \left( -\frac{e_c^\top R^\top P_\beta R}{d} \right) v$は以下のように計算される.
\begin{equation}
\begin{aligned}
\frac{d}{dt} \left( -\frac{e_c^\top R^\top P_\beta R}{d} \right) v &= \underbrace{v^\top R^\top \frac{P_\beta R e_c^\times}{d} \Omega}_{\langle \mathrm{Hess}_p h[v], \Omega \rangle} - \frac{z^\top \dot{P}_\beta}{d} R v - \frac{z^\top P_\beta}{d} R \Omega^\times v - v^\top R^\top \frac{P_\beta z \beta^\top}{d^2} R v \\
&= \langle \mathrm{Hess}_p h[v], \Omega \rangle \\
&\quad \underbrace{-v^\top R^\top \frac{\beta (z^\top P_\beta) + (z^\top \beta) P_\beta + P_\beta z \beta^\top}{d^2} R v - \frac{z^\top P_\beta}{d} R \Omega^\times v}_{\langle \mathrm{Hess}_p h[v], v \rangle} \\
&= \langle \mathrm{Hess}_p h[v], v \rangle + \langle \mathrm{Hess}_p h[v], \Omega \rangle
\label{eq:hess_p_h_v_v}
\end{aligned}
\end{equation}
ここで,$z = R e_c$である.

よって,二次系におけるHOCBFは以下のようになる.
\begin{equation}
\begin{aligned}
&\underbrace{-\frac{e_c^\top R^\top P_\beta R}{d} \dot{v}}_{\langle \mathrm{grad}_p h, \dot{v} \rangle} + \underbrace{v^\top R^\top \frac{P_\beta R e_c^\times}{d} \Omega}_{\langle \mathrm{Hess}_p h[v], \Omega \rangle} \\
&\underbrace{-v^\top R^\top \frac{\beta (z^\top P_\beta) + (z^\top \beta) P_\beta + P_\beta (R e_c) \beta^\top}{d^2} R v - \frac{(R e_c)^\top P_\beta}{d} R \Omega^\times v}_{\langle \mathrm{Hess}_p h[v], v \rangle} \\
&\underbrace{-\beta^\top R e_c^\times \dot{\Omega}}_{\langle \mathrm{grad}_R h, \dot{\Omega} \rangle} + \underbrace{-\beta^\top R \Omega^\times e_c^\times \Omega}_{\langle \mathrm{Hess}_R h[\Omega], \Omega \rangle} + \underbrace{\left( \frac{P_\beta}{d} R v \right)^\top R e_c^\times \Omega}_{\langle \mathrm{Hess}_R h[\Omega], v \rangle} \\
&+ (\gamma_0 + \gamma_1) \left( \underbrace{-\beta^\top R e_c^\times \Omega}_{\langle \mathrm{grad}_R h, \Omega \rangle} + \underbrace{-\frac{e_c^\top R^\top P_\beta R}{d} v}_{\langle \mathrm{grad}_p h, v \rangle} \right) + \gamma_0 \gamma_1 (\beta_l^{\top} R e_c - \cos \Psi_\mathcal{F}) \geq 0
\label{eq:hocbf_constraint}
\end{aligned}
\end{equation}
ここで,$\gamma_0, \gamma_1 > 0$はHOCBFのゲインパラメータである.

$\dot{v}, \dot{\Omega}$を制御入力で表すと,
\begin{equation}
\begin{aligned}
\dot{\Omega} &\simeq J^{-1} \tau \\
\dot{v} &\simeq v^\times \Omega - g R^\top e_z + M^{-1} f
\label{eq:acceleration_control}
\end{aligned}
\end{equation}
となる.これを\Eqref{eq:hocbf_constraint}に代入すると,以下のようになる.
\begin{equation}
\begin{aligned}
&\underbrace{-\beta^\top R e_c^\times J^{-1} \tau}_{\langle \mathrm{grad}_R h, \dot{\Omega} \rangle} + \underbrace{-\frac{e_c^\top R^\top P_\beta R}{d} (v^\times \Omega - g R^\top e_z + M^{-1} f)}_{\langle \mathrm{grad}_p h, \dot{v} \rangle} \\
&+ \underbrace{v^\top R^\top \frac{P_\beta R e_c^\times}{d} \Omega}_{\langle \mathrm{Hess}_p h[v], \Omega \rangle} \\
&+ \underbrace{-v^\top R^\top \frac{\beta (z^\top P_\beta) + (z^\top \beta) P_\beta + P_\beta (R e_c) \beta^\top}{d^2} R v - \frac{(R e_c)^\top P_\beta}{d} R \Omega^\times v}_{\langle \mathrm{Hess}_p h[v], v \rangle} \\
&+ \underbrace{-\beta^\top R \Omega^\times e_c^\times \Omega}_{\langle \mathrm{Hess}_R h[\Omega], \Omega \rangle} + \underbrace{\left( \frac{P_\beta}{d} R v \right)^\top R e_c^\times \Omega}_{\langle \mathrm{Hess}_R h[\Omega], v \rangle} \\
&+ 2 \gamma_0 \left( \underbrace{-\beta^\top R e_c^\times \Omega}_{\langle \mathrm{grad}_R h, \Omega \rangle} + \underbrace{-\frac{e_c^\top R^\top P_\beta R}{d} v}_{\langle \mathrm{grad}_p h, v \rangle} \right) + \gamma_0^2 (\beta_l^{\top} R e_c - \cos \Psi_\mathcal{F}) \geq 0
\label{eq:hocbf_constraint_control}
\end{aligned}
\end{equation}

したがって,HOCBF制約付きQPは以下のように定式化される.
\begin{equation}
\begin{aligned}
\min_{u_k} &\: \frac{1}{2} u_k^\top (A^\top A + B'_1) u_k + (A^\top \tilde{e}_k + B'_2)^\top u_k \\
\mathrm{s.t.} &\: C u_k \leq b
\label{eq:hocbf_qp}
\end{aligned}
\end{equation}
ここで,
\begin{equation}
\begin{aligned}
C &= \begin{bmatrix} \frac{e_c^\top R^\top P_\beta R}{d} M^{-1} & \beta^\top R e_c^\times J^{-1} \end{bmatrix} \in \mathbb{R}^{1 \times 6} \\
b &= \langle \mathrm{Hess}_p h[v], \Omega \rangle + \langle \mathrm{Hess}_p h[v], v \rangle + \langle \mathrm{Hess}_R h[\Omega], v \rangle + \langle \mathrm{Hess}_R h[\Omega], \Omega \rangle \\
&\quad + 2 \gamma_0 (\langle \mathrm{grad}_R h, \Omega \rangle + \langle \mathrm{grad}_p h, v \rangle) + \gamma_0^2 (\beta_l^{\top} R e_c - \cos \Psi_\mathcal{F}) \\
&\quad - \frac{e_c^\top R^\top P_\beta R}{d} (v^\times \Omega - g R^\top e_z)
\label{eq:hocbf_qp_parameters}
\end{aligned}
\end{equation}
である.

\subsection{非ホロノミック系への拡張}

ドローンの非ホロノミック系力学モデルに合わせるため,
\begin{equation}
\begin{aligned}
f_k &\mapsto f e_z, \quad f_k \in \mathbb{R} \\
u_k &= \begin{bmatrix} f_k \\ \tau_k \end{bmatrix} \in \mathbb{R}^4
\label{eq:nonholonomic_control}
\end{aligned}
\end{equation}
とする.非ホロノミック系におけるHOCBF制約付きQPは以下のように定式化される.
\begin{equation}
\begin{aligned}
\min_{u_k} &\: \frac{1}{2} u_k^\top (A^\top A + B'_1) u_k + (A^\top \tilde{e}_k + B'_2)^\top u_k \\
\mathrm{s.t.} &\: C u_k \leq b
\label{eq:nonholonomic_hocbf_qp}
\end{aligned}
\end{equation}
ここで,
\begin{equation}
\begin{aligned}
u_k &= \begin{bmatrix} f_k \\ \tau_k \end{bmatrix} \in \mathbb{R}^4 \\
A &= \begin{bmatrix} -h^2 M^{-1} R_k e_z & h^3 v_k \times J^{-1} \end{bmatrix} \in \mathbb{R}^{3 \times 4} \\
B'_1 &= \begin{bmatrix} M^{-2} b_1 & 0 \\ 0 & (J^{-1})^\top B_2 J^{-1} \end{bmatrix} \in \mathbb{R}^{4 \times 4} \\
B'_2 &= \begin{bmatrix} M^{-1} b_1 A_g^\top e_z \\ 0 \end{bmatrix} \in \mathbb{R}^{4} \\
\tilde{e}_k &= p^d_{k} - p_k - h R_k v_k + h^2 g e_z \\
C &= \begin{bmatrix} \frac{e_c^\top R^\top P_\beta R}{d} e_z M^{-1} & \beta^\top R e_c^\times J^{-1} \end{bmatrix} \in \mathbb{R}^{1 \times 4} \\
b &= \langle \mathrm{Hess}_p h[v], \Omega \rangle + \langle \mathrm{Hess}_p h[v], v \rangle + \langle \mathrm{Hess}_R h[\Omega], v \rangle + \langle \mathrm{Hess}_R h[\Omega], \Omega \rangle \\
&\quad + 2 \gamma_0 (\langle \mathrm{grad}_R h, \Omega \rangle + \langle \mathrm{grad}_p h, v \rangle) + \gamma_0^2 (\beta_l^{\top} R e_c - \cos \Psi_\mathcal{F}) \\
&\quad - \frac{e_c^\top R^\top P_\beta R}{d} (v^\times \Omega - g R^\top e_z)
\label{eq:nonholonomic_hocbf_qp_parameters}
\end{aligned}
\end{equation}
である.

\subsection{複数エージェント・複数特徴点の場合}

以下では非ホロノミック系についてHOCBFの設計を行う.前章と同様に,安全集合を
\begin{equation}
\begin{aligned}
B_{i} &= 1 - q - \eta_{i} \\
\eta_{i} &= \prod_{l \in \mathcal{L}} (1 - \phi_{i}^l)
\label{eq:safe_set_multi_hocbf}
\end{aligned}
\end{equation}
とする.

安全集合の一階微分は前章と同様に
\begin{equation}
\begin{aligned}
\dot{B}_{i} &= -\dot{\eta}_{i} \\
&= -\frac{d}{dt} \prod_{l \in \mathcal{L}} (1 - \phi_{i}^l) \\
&= \sum_{l \in \mathcal{L}} \left( \prod_{k \neq l} (1 - \phi_{i}^k) \right) \dot{\phi}^l_{i} \\
\dot{\phi}^l_{i} &= \left\{ \begin{array}{ll}
\dot{P}_i^l & \mathrm{if} \quad q_l \in \mathcal{C}_i \\
0 & \mathrm{if} \quad q_l \in \mathcal{L} \setminus \mathcal{C}_i
\end{array} \right.
\label{eq:safe_set_derivative_multi_hocbf}
\end{aligned}
\end{equation}
となる.

安全集合の二階微分は以下のように計算される.
\begin{equation}
\begin{aligned}
\ddot{B}_i &= \frac{d}{dt} \dot{B}_i \\
&= \frac{d}{dt} \left( \sum_{l \in \mathcal{L}} \left( \prod_{k \neq l} (1 - \phi_{i}^k) \right) \dot{\phi}^l_{i} \right) \\
&= \sum_{l \in \mathcal{L}} \frac{d}{dt} \left( \left( \prod_{k \neq l} (1 - \phi_{i}^k) \right) \dot{\phi}^l_{i} \right) \\
&= \sum_{l \in \mathcal{L}} \left( \frac{d}{dt} \left( \prod_{k \neq l} (1 - \phi_{i}^k) \right) \dot{\phi}^l_{i} + \left( \prod_{k \neq l} (1 - \phi_{i}^k) \right) \ddot{\phi}^l_{i} \right) \\
&= \sum_{l \in \mathcal{L}} \left( -\sum_{j \neq l} \left( \prod_{m \neq j, l} (1 - \phi_{i}^m) \right) \dot{\phi}_i^j \dot{\phi}^l_{i} + \left( \prod_{k \neq l} (1 - \phi_{i}^k) \right) \ddot{\phi}^l_{i} \right)
\label{eq:safe_set_second_derivative_multi_hocbf}
\end{aligned}
\end{equation}

ここで,$\ddot{\phi}^l_{i}$は$P_i^l$の二階微分であり,以下のように計算できる.
\begin{equation}
\begin{aligned}
\ddot{\phi}^l_{i} &= \left\{ \begin{array}{ll}
\ddot{P}_i^l & \mathrm{if} \quad q_l \in \mathcal{C}_i \\
0 & \mathrm{if} \quad q_l \in \mathcal{L} \setminus \mathcal{C}_i
\end{array} \right. \\
\ddot{P}_i^l &= \frac{d}{dt} \dot{P}_i^l \\
&= \frac{d}{dt} \langle \mathrm{grad}\:P_i^l, \xi_W \rangle \\
&= \langle \mathrm{Hess}\:P_i^l[\xi_W], \xi_W \rangle + \langle \mathrm{grad}\:P_i^l, \dot{\xi}_W \rangle
\label{eq:probability_second_derivative}
\end{aligned}
\end{equation}

ここで,$\mathrm{Hess}\:P_i^l$は$P_i^l$のヘッシアン行列であり,$\dot{\xi}_W$は制御入力に依存する項である.$\mathrm{Hess}\:P_i^l$は以下のように分解できる.
\begin{equation}
\begin{aligned}
\langle \mathrm{Hess}\:P_i^l[\xi_W], \xi_W \rangle &= \langle \mathrm{Hess}_p\:P_i^l[v], v \rangle + \langle \mathrm{Hess}_p\:P_i^l[v], \omega \rangle + \langle \mathrm{Hess}_R\:P_i^l[\omega], v \rangle + \langle \mathrm{Hess}_R\:P_i^l[\omega], \omega \rangle
\label{eq:hessian_decomposition}
\end{aligned}
\end{equation}

また,$\langle \mathrm{grad}\:P_i^l, \dot{\xi}_W \rangle$は以下のように分解できる.
\begin{equation}
\begin{aligned}
\langle \mathrm{grad}\:P_i^l, \dot{\xi}_W \rangle &= \langle \mathrm{grad}_p\:P_i^l, \dot{v} \rangle + \langle \mathrm{grad}_R\:P_i^l, \dot{\omega} \rangle
\label{eq:grad_dot_xi}
\end{aligned}
\end{equation}

これらを用いて,HOCBFの制約は以下のように表される.
\begin{equation}
\begin{aligned}
\ddot{B}_i + \gamma_1 \dot{B}_i + \gamma_0 B_i \geq 0
\label{eq:hocbf_constraint_multi}
\end{aligned}
\end{equation}

ここで,$\gamma_0, \gamma_1 > 0$は正の定数である.この制約を展開すると:
\begin{equation}
\begin{aligned}
&\sum_{l \in \mathcal{L}} \left( -\sum_{j \neq l} \left( \prod_{m \neq j, l} (1 - \phi_{i}^m) \right) \dot{\phi}_i^j \dot{\phi}^l_{i} + \left( \prod_{k \neq l} (1 - \phi_{i}^k) \right) \ddot{\phi}^l_{i} \right) \\
&+ \gamma_1 \sum_{l \in \mathcal{L}} \left( \prod_{k \neq l} (1 - \phi_{i}^k) \right) \dot{\phi}^l_{i} \\
&+ \gamma_0 (1 - q - \prod_{l \in \mathcal{L}} (1 - \phi_{i}^l)) \geq 0
\label{eq:hocbf_constraint_multi_expanded}
\end{aligned}
\end{equation}

制御入力$u_k = [f_k, \tau_k]^\top$に依存する項は$\ddot{\phi}^l_{i}$の中の$\langle \mathrm{grad}_p\:P_i^l, \dot{v} \rangle$と$\langle \mathrm{grad}_R\:P_i^l, \dot{\omega} \rangle$である.これらを制御入力について整理すると:
\begin{equation}
\begin{aligned}
\langle \mathrm{grad}_p\:P_i^l, \dot{v} \rangle &= \frac{1}{1 - \cos \Psi_\mathcal{F}} \left( -\frac{e_c^\top R_i^\top P_{\beta_l}}{d_{i,l}} \right) \dot{v} \\
\langle \mathrm{grad}_R\:P_i^l, \dot{\omega} \rangle &= \frac{1}{1 - \cos \Psi_\mathcal{F}} (-\beta_l^\top(p_i) R_i [e_c]_\times) \dot{\omega}
\label{eq:grad_dot_v_omega}
\end{aligned}
\end{equation}

$\dot{v}$と$\dot{\omega}$を制御入力$u_k = [f_k, \tau_k]^\top$で表すと:
\begin{equation}
\begin{aligned}
\dot{v} &= v^\times \Omega - g R^\top e_z + M^{-1} f \\
\dot{\omega} &= J^{-1} \tau
\label{eq:acceleration_control_multi}
\end{aligned}
\end{equation}

これらを代入して整理すると,以下のような制約付きQPが得られる.
\begin{equation}
\begin{aligned}
\min_{u_k} &\: \frac{1}{2} u_k^\top (A^\top A + B'_1) u_k + (A^\top \tilde{e}_k + B'_2)^\top u_k \\
\mathrm{s.t.} &\: C_{\mathrm{multi}} u_k \geq b_{\mathrm{multi}}
\label{eq:hocbf_qp_multi}
\end{aligned}
\end{equation}

ここで,
\begin{equation}
\begin{aligned}
C_{\mathrm{multi}} &= \begin{bmatrix}
\sum_{l \in \mathcal{L} \cap \mathcal{C}_i} \left( \prod_{k \neq l} (1 - \phi_{i}^k) \right) \frac{e_c^\top R_i^\top P_{\beta_l}}{(1 - \cos \Psi_\mathcal{F}) d_{i,l}} e_z M^{-1} & \sum_{l \in \mathcal{L} \cap \mathcal{C}_i} \left( \prod_{k \neq l} (1 - \phi_{i}^k) \right) \frac{\beta_l^\top(p_i) R_i [e_c]_\times}{1 - \cos \Psi_\mathcal{F}} J^{-1}
\end{bmatrix} \\
b_{\mathrm{multi}} &= -\sum_{l \in \mathcal{L}} \left( -\sum_{j \neq l} \left( \prod_{m \neq j, l} (1 - \phi_{i}^m) \right) \dot{\phi}_i^j \dot{\phi}^l_{i} + \left( \prod_{k \neq l} (1 - \phi_{i}^k) \right) (\langle \mathrm{Hess}\:P_i^l[\xi_W], \xi_W \rangle) \right) \\
&\quad - \gamma_1 \sum_{l \in \mathcal{L}} \left( \prod_{k \neq l} (1 - \phi_{i}^k) \right) \dot{\phi}^l_{i} \\
&\quad - \gamma_0 (1 - q - \prod_{l \in \mathcal{L}} (1 - \phi_{i}^l)) \\
&\quad + \sum_{l \in \mathcal{L} \cap \mathcal{C}_i} \left( \prod_{k \neq l} (1 - \phi_{i}^k) \right) \frac{e_c^\top R_i^\top P_{\beta_l}}{(1 - \cos \Psi_\mathcal{F}) d_{i,l}} (v^\times \Omega - g R^\top e_z)
\label{eq:hocbf_qp_multi_parameters}
\end{aligned}
\end{equation}

\subsection{複数エージェントが共通の特徴点を追従する場合(HOCBF)}

複数のエージェントが共通の特徴点を追従する場合,安全集合を以下のように定義する.
\begin{equation}
\begin{aligned}
B_{ij} &= 1 - q - \eta_{ij} \\
\eta_{ij} &= \prod_{l \in \mathcal{L}} (1 - \phi_{ij}^l)
\label{eq:safe_set_edge_hocbf}
\end{aligned}
\end{equation}

ここで,$\phi_{ij}^l$は特徴点$q_l \in \mathcal{L}$によってエッジ$(i,j) \in \mathcal{E}$における推定が成り立っている確率であり,以下のように定義される.
\begin{equation}
\begin{aligned}
\phi_{ij}^l &= \left\{ \begin{array}{ll}
P_i^l P_j^l & \mathrm{if} \quad q_l \in \mathcal{C}_i \cap \mathcal{C}_j \\
0 & \mathrm{if} \quad q_l \in \mathcal{L} \setminus (\mathcal{C}_i \cap \mathcal{C}_j)
\end{array} \right. \\
\mathrm{where} \quad P_i^l &= \frac{\beta_l^\top(p_i) R_i e_c - \cos \Psi_\mathcal{F}}{1 - \cos \Psi_\mathcal{F}}
\label{eq:probability_edge_hocbf}
\end{aligned}
\end{equation}

安全集合$B_{ij}$の時間微分は以下のように計算される.
\begin{equation}
\begin{aligned}
\dot{B}_{ij} &= -\dot{\eta}_{ij} \\
&= -\frac{d}{dt} \prod_{l \in \mathcal{L}} (1 - \phi_{ij}^l) \\
&= \sum_{l \in \mathcal{L}} \left( \prod_{k \neq l} (1 - \phi_{ij}^k) \right) \dot{\phi}^l_{ij} \\
\dot{\phi}^l_{ij} &= \left\{ \begin{array}{ll}
\dot{P}_i^l P_j^l + P_i^l \dot{P}_j^l & \mathrm{if} \quad q_l \in \mathcal{C}_i \cap \mathcal{C}_j \\
0 & \mathrm{if} \quad q_l \in \mathcal{L} \setminus (\mathcal{C}_i \cap \mathcal{C}_j)
\end{array} \right.
\label{eq:safe_set_derivative_edge_hocbf}
\end{aligned}
\end{equation}

安全集合$B_{ij}$の二階微分を計算するために,$\dot{B}_{ij}$の時間微分を計算する.
\begin{equation}
\begin{aligned}
\ddot{B}_{ij} &= \frac{d}{dt} \dot{B}_{ij} \\
&= \frac{d}{dt} \left( \sum_{l \in \mathcal{L}} \left( \prod_{k \neq l} (1 - \phi_{ij}^k) \right) \dot{\phi}^l_{ij} \right) \\
&= \sum_{l \in \mathcal{L}} \frac{d}{dt} \left( \left( \prod_{k \neq l} (1 - \phi_{ij}^k) \right) \dot{\phi}^l_{ij} \right) \\
&= \sum_{l \in \mathcal{L}} \left( \frac{d}{dt} \left( \prod_{k \neq l} (1 - \phi_{ij}^k) \right) \dot{\phi}^l_{ij} + \left( \prod_{k \neq l} (1 - \phi_{ij}^k) \right) \ddot{\phi}^l_{ij} \right) \\
&= \sum_{l \in \mathcal{L}} \left( -\sum_{j \neq l} \left( \prod_{m \neq j, l} (1 - \phi_{ij}^m) \right) \dot{\phi}_{ij}^j \dot{\phi}^l_{ij} + \left( \prod_{k \neq l} (1 - \phi_{ij}^k) \right) \ddot{\phi}^l_{ij} \right)
\label{eq:safe_set_second_derivative_edge_hocbf}
\end{aligned}
\end{equation}

ここで,$\ddot{\phi}^l_{ij}$は$\phi_{ij}^l$の二階微分であり,以下のように計算できる.
\begin{equation}
\begin{aligned}
\ddot{\phi}^l_{ij} &= \left\{ \begin{array}{ll}
\ddot{P}_i^l P_j^l + 2 \dot{P}_i^l \dot{P}_j^l + P_i^l \ddot{P}_j^l & \mathrm{if} \quad q_l \in \mathcal{C}_i \cap \mathcal{C}_j \\
0 & \mathrm{if} \quad q_l \in \mathcal{L} \setminus (\mathcal{C}_i \cap \mathcal{C}_j)
\end{array} \right.
\label{eq:probability_second_derivative_edge}
\end{aligned}
\end{equation}

$\ddot{P}_i^l$と$\ddot{P}_j^l$は$P_i^l$と$P_j^l$の二階微分であり,前述の方法で計算できる.

これらを用いて,HOCBFの制約は以下のように表される.
\begin{equation}
\begin{aligned}
\ddot{B}_{ij} + \gamma_1 \dot{B}_{ij} + \gamma_0 B_{ij} \geq 0
\label{eq:hocbf_constraint_edge}
\end{aligned}
\end{equation}

ここで,$\gamma_0, \gamma_1 > 0$は正の定数である.この制約を展開すると:
\begin{equation}
\begin{aligned}
&\sum_{l \in \mathcal{L}} \left( -\sum_{j \neq l} \left( \prod_{m \neq j, l} (1 - \phi_{ij}^m) \right) \dot{\phi}_{ij}^j \dot{\phi}^l_{ij} + \left( \prod_{k \neq l} (1 - \phi_{ij}^k) \right) \ddot{\phi}^l_{ij} \right) \\
&+ \gamma_1 \sum_{l \in \mathcal{L}} \left( \prod_{k \neq l} (1 - \phi_{ij}^k) \right) \dot{\phi}^l_{ij} \\
&+ \gamma_0 (1 - q - \prod_{l \in \mathcal{L}} (1 - \phi_{ij}^l)) \geq 0
\label{eq:hocbf_constraint_edge_expanded}
\end{aligned}
\end{equation}

制御入力$u_i = [f_i, \tau_i]^\top$と$u_j = [f_j, \tau_j]^\top$に依存する項は$\ddot{\phi}^l_{ij}$の中の$\langle \mathrm{grad}_p\:P_i^l, \dot{v}_i \rangle$,$\langle \mathrm{grad}_R\:P_i^l, \dot{\omega}_i \rangle$,$\langle \mathrm{grad}_p\:P_j^l, \dot{v}_j \rangle$,$\langle \mathrm{grad}_R\:P_j^l, \dot{\omega}_j \rangle$である.

これらを制御入力について整理し,前述の方法と同様にQPを定式化することで,複数エージェントが共通の特徴点を追従するためのHOCBF制約付きQPが得られる.

本章では,SE(3)上の剛体運動に対して,視野共有を保証するための高次制御バリア関数(HOCBF)を設計し,それに基づく制御則を提案した.これにより,実際のドローンダイナミクスを考慮した共有視野保証が可能となる.

\section{Simulation Results}

本章では,提案手法の有効性を検証するためのシミュレーション結果を示す.まず,単一特徴点追従のシミュレーション結果を示し,次に複数特徴点追従,そして複数エージェントの共通特徴点追従の結果を示す.最後に,分散実装の性能評価を行う.

\subsection{シミュレーション設定}

シミュレーションでは,以下のパラメータを用いた.
\begin{itemize}
    \item ドローンの質量: $m = 1.0$ kg
    \item 慣性モーメント: $J = \mathrm{diag}(0.01, 0.01, 0.02)$ kg$\cdot$m$^2$
    \item カメラの視野角: $\Psi_\mathcal{F} = 60^\circ$
    \item 離散時間ステップ: $h = 0.01$ s
    \item CBFのゲインパラメータ: $\gamma = 1.0$, $\gamma_0 = 1.0$, $\gamma_1 = 2.0$
    \item 確率パラメータ: $q = 0.8$
    \item 目標位置への追従重み: $Q_1 = \mathrm{diag}(1.0, 1.0, 1.0)$
    \item 制御入力の重み: $Q_2 = \mathrm{diag}(0.1, 0.1, 0.1, 0.1, 0.1, 0.1)$
\end{itemize}

環境内には複数の特徴点を配置し,エージェントは初期位置から目標位置に向かって移動する.シミュレーションは,Python環境で実装し,QP問題の解法にはCVXPYを用いた.

\subsection{単一特徴点追従のシミュレーション}

まず,単一のエージェントが単一の特徴点を視野内に保持しながら目標位置に移動するシミュレーションを行った.図\ref{fig:single_feature_trajectory}に,エージェントの軌跡と特徴点の位置を示す.

\begin{figure}[htbp]
    \centering
    % \includegraphics[width=0.8\linewidth]{Fig/single_feature_trajectory.eps}
    \caption{単一特徴点追従におけるエージェントの軌跡.青線はエージェントの軌跡,赤点は特徴点の位置,緑点は目標位置を表す.}
    \label{fig:single_feature_trajectory}
\end{figure}

図\ref{fig:single_feature_cbf_value}に,シミュレーション中のCBF値の推移を示す.CBF値は常に正の値を保っており,安全制約が満たされていることがわかる.

\begin{figure}[htbp]
    \centering
    % \includegraphics[width=0.8\linewidth]{Fig/single_feature_cbf_value.eps}
    \caption{単一特徴点追従におけるCBF値の推移.CBF値が常に正であり,安全制約が満たされていることを示している.}
    \label{fig:single_feature_cbf_value}
\end{figure}

図\ref{fig:single_feature_control_input}に,制御入力の推移を示す.制御入力は滑らかに変化しており,実機への適用が可能であることがわかる.

\begin{figure}[htbp]
    \centering
    % \includegraphics[width=0.8\linewidth]{Fig/single_feature_control_input.eps}
    \caption{単一特徴点追従における制御入力の推移.(a)並進力,(b)トルク.}
    \label{fig:single_feature_control_input}
\end{figure}

\subsection{複数特徴点追従のシミュレーション}

次に,単一のエージェントが複数の特徴点を視野内に保持しながら目標位置に移動するシミュレーションを行った.図\ref{fig:multi_feature_trajectory}に,エージェントの軌跡と特徴点の位置を示す.

\begin{figure}[htbp]
    \centering
    % \includegraphics[width=0.8\linewidth]{Fig/multi_feature_trajectory.eps}
    \caption{複数特徴点追従におけるエージェントの軌跡.青線はエージェントの軌跡,赤点は特徴点の位置,緑点は目標位置を表す.}
    \label{fig:multi_feature_trajectory}
\end{figure}

図\ref{fig:multi_feature_probability}に,各特徴点の視野内確率$P_i^l$の推移を示す.確率値は0から1の間で変化しており,エージェントの移動に伴って特徴点が視野内に入ったり出たりする様子がわかる.

\begin{figure}[htbp]
    \centering
    % \includegraphics[width=0.8\linewidth]{Fig/multi_feature_probability.eps}
    \caption{複数特徴点追従における各特徴点の視野内確率$P_i^l$の推移.異なる色の線は異なる特徴点に対応している.}
    \label{fig:multi_feature_probability}
\end{figure}

図\ref{fig:multi_feature_cbf_value}に,シミュレーション中のCBF値の推移を示す.CBF値は常に正の値を保っており,安全制約が満たされていることがわかる.

\begin{figure}[htbp]
    \centering
    % \includegraphics[width=0.8\linewidth]{Fig/multi_feature_cbf_value.eps}
    \caption{複数特徴点追従におけるCBF値の推移.CBF値が常に正であり,安全制約が満たされていることを示している.}
    \label{fig:multi_feature_cbf_value}
\end{figure}

\subsection{複数エージェントの共通特徴点追従}

次に,複数のエージェントが共通の特徴点を視野内に保持しながら目標位置に移動するシミュレーションを行った.図\ref{fig:multi_agent_trajectory}に,エージェントの軌跡と特徴点の位置を示す.

\begin{figure}[htbp]
    \centering
    % \includegraphics[width=0.8\linewidth]{Fig/multi_agent_trajectory.eps}
    \caption{複数エージェントの共通特徴点追従におけるエージェントの軌跡.青線と緑線はそれぞれエージェント1と2の軌跡,赤点は特徴点の位置,青点と緑点はそれぞれエージェント1と2の目標位置を表す.}
    \label{fig:multi_agent_trajectory}
\end{figure}

図\ref{fig:multi_agent_cbf_value}に,シミュレーション中のCBF値の推移を示す.CBF値は常に正の値を保っており,安全制約が満たされていることがわかる.

\begin{figure}[htbp]
    \centering
    % \includegraphics[width=0.8\linewidth]{Fig/multi_agent_cbf_value.eps}
    \caption{複数エージェントの共通特徴点追従におけるCBF値の推移.CBF値が常に正であり,安全制約が満たされていることを示している.}
    \label{fig:multi_agent_cbf_value}
\end{figure}

図\ref{fig:multi_agent_shared_features}に,共有視野内の特徴点数の推移を示す.共有視野内の特徴点数は常に設定した最小数$m$以上を保っており,共有視野が保証されていることがわかる.

\begin{figure}[htbp]
    \centering
    % \includegraphics[width=0.8\linewidth]{Fig/multi_agent_shared_features.eps}
    \caption{複数エージェントの共通特徴点追従における共有視野内の特徴点数の推移.赤線は設定した最小数$m$を表す.}
    \label{fig:multi_agent_shared_features}
\end{figure}

\subsection{分散実装の性能評価}

最後に,分散実装の性能評価を行った.図\ref{fig:distributed_convergence}に,分散最適化アルゴリズム(PDMM)の収束性を示す.

\begin{figure}[htbp]
    \centering
    % \includegraphics[width=0.8\linewidth]{Fig/distributed_convergence.eps}
    \caption{分散最適化アルゴリズム(PDMM)の収束性.(a)目的関数値の推移,(b)制約違反の推移.}
    \label{fig:distributed_convergence}
\end{figure}

図\ref{fig:distributed_vs_centralized}に,分散実装と中央集権的実装の比較を示す.分散実装は中央集権的実装と比較して,計算時間が短く,スケーラビリティに優れていることがわかる.

\begin{figure}[htbp]
    \centering
    % \includegraphics[width=0.8\linewidth]{Fig/distributed_vs_centralized.eps}
    \caption{分散実装と中央集権的実装の比較.(a)計算時間,(b)通信量.}
    \label{fig:distributed_vs_centralized}
\end{figure}

\subsection{二次系システムのためのHOCBFの評価}

最後に,二次系システムのためのHOCBFの評価を行った.図\ref{fig:hocbf_trajectory}に,HOCBFを用いたエージェントの軌跡を示す.

\begin{figure}[htbp]
    \centering
    % \includegraphics[width=0.8\linewidth]{Fig/hocbf_trajectory.eps}
    \caption{HOCBFを用いたエージェントの軌跡.青線はエージェントの軌跡,赤点は特徴点の位置,緑点は目標位置を表す.}
    \label{fig:hocbf_trajectory}
\end{figure}

図\ref{fig:hocbf_cbf_value}に,シミュレーション中のHOCBF値の推移を示す.HOCBF値は常に正の値を保っており,安全制約が満たされていることがわかる.

\begin{figure}[htbp]
    \centering
    % \includegraphics[width=0.8\linewidth]{Fig/hocbf_cbf_value.eps}
    \caption{HOCBFを用いたシミュレーションにおけるHOCBF値の推移.HOCBF値が常に正であり,安全制約が満たされていることを示している.}
    \label{fig:hocbf_cbf_value}
\end{figure}

図\ref{fig:hocbf_vs_cbf}に,HOCBFとCBFの比較を示す.HOCBFはCBFと比較して,より滑らかな制御入力を生成し,安全制約の満足度も高いことがわかる.

\begin{figure}[htbp]
    \centering
    % \includegraphics[width=0.8\linewidth]{Fig/hocbf_vs_cbf.eps}
    \caption{HOCBFとCBFの比較.(a)制御入力の滑らかさ,(b)安全制約の満足度.}
    \label{fig:hocbf_vs_cbf}
\end{figure}

\subsection{考察}

シミュレーション結果から,以下のことがわかった.

\begin{enumerate}
    \item 提案手法は,単一特徴点追従,複数特徴点追従,複数エージェントの共通特徴点追従のいずれの場合も,安全制約を満たしながら目標位置への追従が可能である.
    \item 確率的アプローチにより,複数の特徴点を視野内に保持する制約を滑らかに表現でき,最適化問題の解法が容易になる.
    \item 分散実装は,中央集権的実装と比較して計算効率が高く,スケーラビリティに優れている.
    \item HOCBFは,二次系システムに対して滑らかな制御入力を生成し,安全制約の満足度も高い.
\end{enumerate}

これらの結果から,提案手法は実機への適用が期待できる.特に,複数のドローンが協調して自己位置推定を行う場合に,視野共有を保証することで推定精度の向上が期待できる.

\section{Conclusion}

本論文では,SE(3)上における協調自己位置推定のための視野共有を保証する分散型CBFを提案した.提案手法は,単一特徴点追従,複数特徴点追従,複数エージェントの共通特徴点追従,そして二次系システムのための高次CBFへと段階的に拡張され,それぞれの場合について理論的な解析と定式化を行った.

\subsection{研究成果のまとめ}

本研究の主な成果は以下の通りである.

\begin{enumerate}
    \item SE(3)上での共有視野保証:従来研究は主に平面上の(SE(2))あるいは単眼視野の問題に限定されていたが,本研究では3次元の複雑なダイナミクス下での共有視野保証を実現した.エージェント位置を$T_i=(p_i, R_i)\in \mathrm{SE}(3)$,環境内の特徴点を$q_l\in \mathcal{L}$として,視野内条件$\beta_l^{\top}(p_i)R_ie_c-\cos\Psi_\mathcal{F}>0$に基づく安全集合を定義し,CBFを設計した.
    
    \item 特徴点に基づく確率的可視性制約とCBFの適用:各エージェントが観測する特徴点に基づき,その可視性を確率的に評価した上で,制御バリア関数(CBF)に組み込み,常時高い確率で共有視野が確保されるよう制御入力を設計した.特に,複数の特徴点を追従する場合に,安全集合$B_{i}=1-q-\prod_{l\in\mathcal{L}}(1-\phi_{i}^l)$を定義し,確率$q$以上で特徴点が視野内に保持されるよう制約を設計した.
    
    \item 非ホロノミックなドローンダイナミクスへの対応と分散最適化:機体の並進および回転運動を同時に考慮するSE(3)上の非ホロノミックドローンモデルに対して,高次制約も扱える制御バリア関数(HOCBF)を導入し,各エージェントが局所的な情報交換を通じて分散最適化アルゴリズム(PDMM)により制御解を求める枠組みを提案した.これにより,リアルタイム性とスケーラビリティの両立を実現した.
    
    \item 二次系システムのための高次CBF:実機への適用を考慮し,ドローンの二次系ダイナミクスに対応するHOCBFを設計した.SE(3)における離散ダイナミクスを導出し,ホロノミック系と非ホロノミック系それぞれに対するQP定式化を行った.これにより,より実際的なドローンモデルに対しても視野共有保証が可能となった.
\end{enumerate}

シミュレーション結果から,提案手法は安全制約を満たしながら目標位置への追従が可能であることが示された.特に,確率的アプローチにより複数の特徴点を視野内に保持する制約を滑らかに表現でき,分散実装により計算効率とスケーラビリティが向上することが確認された.また,HOCBFは二次系システムに対して滑らかな制御入力を生成し,安全制約の満足度も高いことが示された.

\subsection{今後の課題}

本研究の今後の課題として,以下の点が挙げられる.

\begin{enumerate}
    \item 実機実験による検証:本研究ではシミュレーションによる検証を行ったが,実機実験による検証は今後の課題である.特に,センサノイズや通信遅延などの実環境での不確実性に対するロバスト性の評価が必要である.
    
    \item 自己位置推定との統合:本研究では視野共有を保証するCBFを設計したが,実際の自己位置推定アルゴリズム(CoVINSなど)との統合は今後の課題である.視野共有保証と自己位置推定の精度向上の関係を定量的に評価することが重要である.
    
    \item 障害物回避との統合:実環境では障害物が存在するため,視野共有保証と障害物回避を同時に満たす制御則の設計が必要である.複数の安全制約を統合するための手法の開発が課題となる.
    
    \item スケーラビリティの向上:より多数のエージェントに対応するため,分散アルゴリズムのさらなる効率化や,グラフ構造を考慮した通信トポロジの最適化が課題である.
    
    \item 動的環境への対応:本研究では静的な環境を仮定したが,動的な環境(移動する特徴点や障害物)に対応するための拡張が必要である.特に,予測に基づく制御や適応的なCBF設計が課題となる.
\end{enumerate}

これらの課題に取り組むことで,より実用的な視野共有保証手法の開発が期待される.特に,自己位置推定との統合により,マルチロボットシステムの協調自己位置推定の精度向上に貢献できると考えられる.

本研究の成果は,ドローンの協調自己位置推定だけでなく,監視・捜索・協調輸送などの様々なマルチロボットタスクにおいて,視野共有を保証するための基盤技術として応用可能である.今後は,より複雑な環境や多様なタスクに対応するための拡張を進めていく予定である.

\begin{thebibliography}{99}
\bibitem{Sample2025} 著者名, ``論文タイトル,'' {\it ジャーナル名}, Vol. 1, No. 1, pp. 1-10, 2025.

\bibitem{LaTeX2025} LaTeX Project Team, ``LaTeX: A Document Preparation System,'' {\it Technical Report}, 2025.

\bibitem{Science2025} 科学太郎, 技術花子, ``科学技術論文の書き方,'' {\it 科学技術ジャーナル}, Vol. 10, No. 5, pp. 123-145, 2025.
\end{thebibliography}


\end{document}
