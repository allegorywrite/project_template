\section{Conclusion}

本論文では,SE(3)上における協調自己位置推定のための視野共有を保証する分散型CBFを提案した.提案手法は,単一特徴点追従,複数特徴点追従,複数エージェントの共通特徴点追従,そして二次系システムのための高次CBFへと段階的に拡張され,それぞれの場合について理論的な解析と定式化を行った.

\subsection{研究成果のまとめ}

本研究の主な成果は以下の通りである.

\begin{enumerate}
    \item SE(3)上での共有視野保証:従来研究は主に平面上の(SE(2))あるいは単眼視野の問題に限定されていたが,本研究では3次元の複雑なダイナミクス下での共有視野保証を実現した.エージェント位置を$T_i=(p_i, R_i)\in \mathrm{SE}(3)$,環境内の特徴点を$q_l\in \mathcal{L}$として,視野内条件$\beta_l^{\top}(p_i)R_ie_c-\cos\Psi_\mathcal{F}>0$に基づく安全集合を定義し,CBFを設計した.
    
    \item 特徴点に基づく確率的可視性制約とCBFの適用:各エージェントが観測する特徴点に基づき,その可視性を確率的に評価した上で,制御バリア関数(CBF)に組み込み,常時高い確率で共有視野が確保されるよう制御入力を設計した.特に,複数の特徴点を追従する場合に,安全集合$B_{i}=1-q-\prod_{l\in\mathcal{L}}(1-\phi_{i}^l)$を定義し,確率$q$以上で特徴点が視野内に保持されるよう制約を設計した.
    
    \item 非ホロノミックなドローンダイナミクスへの対応と分散最適化:機体の並進および回転運動を同時に考慮するSE(3)上の非ホロノミックドローンモデルに対して,高次制約も扱える制御バリア関数(HOCBF)を導入し,各エージェントが局所的な情報交換を通じて分散最適化アルゴリズム(PDMM)により制御解を求める枠組みを提案した.これにより,リアルタイム性とスケーラビリティの両立を実現した.
    
    \item 二次系システムのための高次CBF:実機への適用を考慮し,ドローンの二次系ダイナミクスに対応するHOCBFを設計した.SE(3)における離散ダイナミクスを導出し,ホロノミック系と非ホロノミック系それぞれに対するQP定式化を行った.これにより,より実際的なドローンモデルに対しても視野共有保証が可能となった.
\end{enumerate}

シミュレーション結果から,提案手法は安全制約を満たしながら目標位置への追従が可能であることが示された.特に,確率的アプローチにより複数の特徴点を視野内に保持する制約を滑らかに表現でき,分散実装により計算効率とスケーラビリティが向上することが確認された.また,HOCBFは二次系システムに対して滑らかな制御入力を生成し,安全制約の満足度も高いことが示された.

\subsection{今後の課題}

本研究の今後の課題として,以下の点が挙げられる.

\begin{enumerate}
    \item 実機実験による検証:本研究ではシミュレーションによる検証を行ったが,実機実験による検証は今後の課題である.特に,センサノイズや通信遅延などの実環境での不確実性に対するロバスト性の評価が必要である.
    
    \item 自己位置推定との統合:本研究では視野共有を保証するCBFを設計したが,実際の自己位置推定アルゴリズム(CoVINSなど)との統合は今後の課題である.視野共有保証と自己位置推定の精度向上の関係を定量的に評価することが重要である.
    
    \item 障害物回避との統合:実環境では障害物が存在するため,視野共有保証と障害物回避を同時に満たす制御則の設計が必要である.複数の安全制約を統合するための手法の開発が課題となる.
    
    \item スケーラビリティの向上:より多数のエージェントに対応するため,分散アルゴリズムのさらなる効率化や,グラフ構造を考慮した通信トポロジの最適化が課題である.
    
    \item 動的環境への対応:本研究では静的な環境を仮定したが,動的な環境(移動する特徴点や障害物)に対応するための拡張が必要である.特に,予測に基づく制御や適応的なCBF設計が課題となる.
\end{enumerate}

これらの課題に取り組むことで,より実用的な視野共有保証手法の開発が期待される.特に,自己位置推定との統合により,マルチロボットシステムの協調自己位置推定の精度向上に貢献できると考えられる.

本研究の成果は,ドローンの協調自己位置推定だけでなく,監視・捜索・協調輸送などの様々なマルチロボットタスクにおいて,視野共有を保証するための基盤技術として応用可能である.今後は,より複雑な環境や多様なタスクに対応するための拡張を進めていく予定である.
