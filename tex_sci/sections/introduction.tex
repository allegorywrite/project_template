\section{Introduction}

マルチロボットシステムにおいて,各ロボットがセンサ情報や視界を共有することにより,監視・捜索・協調輸送などのタスクを効率的に遂行できることが知られている.特にカメラを用いた視覚協調の場合,各ロボットの共有視野(Common Field-of-View, CoFoV)が不可欠であり,たとえば複数ドローンが異なる角度から同一対象を観測できることや,通信の見通し線(LOS)維持が求められる\cite{Panagou2017}.しかし,ロボットは視野角が限られているため,視界共有を保証するための制御技術の開発が急務である.

マルチエージェントVSLAMでは,各エージェントが撮影した画像から抽出される局所特徴に基づき自己局在を行い,エージェント間で特徴マッチングにより相対位置を推定する.従来のORBやSIFTに代わり,SuperPointのような学習型局所特徴検出・記述器\cite{DeTone2018}や,NetVLADによるグローバル記述子\cite{Arandjelovic2016}が高いロバスト性を示し,ループ検出やキーフレームマッチングに貢献している.これらの技術は,マルチエージェント間の地図統合とループ閉じの精度向上に直結する.

画像特徴量のマッチングを成功させるためには,各エージェントが共通のランドマークを観測できるよう,カメラの視野錐台の幾何学的重複が必要である.視野重複が存在すれば,エージェント間でのループ検出が可能となり,その結果として各ロボットの地図統合が実現される\cite{Zhang2022}.しかし,現状の協調SLAMは,キーフレームの受動的な共有と画像類似度評価に依存しており,視野重複が偶発的に発生しなければマップ統合が成立しないリスクがある.

この問題に対して,近年ではロボットの運動制御に視野保持の制約を組み込み,常にあるいは定期的にカメラが共通領域を撮像するよう制御するアプローチが提案され始めている.具体的には,制御バリア関数(Control Barrier Function, CBF)を用いて「他エージェントまたは環境ランドマークを視界に捉える」という安全制約を実現することで,受動的手法の限界を克服しようとする試みである\cite{Trimarchi2025}.

本研究では,CoVINSにおける自己位置推定の精度向上を目的として,SE(3)上における協調自己位置推定のための視野共有を保証する分散型CBFを提案する.CoVINSではsuperpoint特徴量などの画像特徴量のマッチングにより,自己位置推定に関する最適化問題のfactorを得ている.特徴量をマッチングするにはエージェントの視野錐台が重なっている必要がある.single agent問題に関してはstereo cameraの相対位置をconstかつgivenとして最適化問題に組み込み,multi agent問題に関しては視野錐台のoverrideは不可知であるために画像全体特徴量の類似度の一致などによってpassiveなevent trigger型としてアルゴリズムが構築されている.

しかし,agent1とagent2が視野錐台を交差させ続けるための制御則(CBF等)に基づいてactive perceptionを行う場合,multi agentの自己位置推定においてinter-agentな特徴量のマッチング及び推定問題はエージェントをまたいだカメラ間の相対位置をgiven,もしくは最適化すべき双対変数として複数のエージェント位置を同時最適化できるはずである.さらに,active perceptionの枠組みで考えれば,CBFを用いた最適制御問題と自己位置推定問題も同一の目的関数の最小化問題として扱うことができるはずである.

本論文の主な貢献は以下の通りである:

\begin{enumerate}
    \item SE(3)上での共有視野保証:従来研究は主に平面上の(SE(2))あるいは単眼視野の問題に限定されていたが,本研究では3次元の複雑なダイナミクス下での共有視野保証を実現する.
    \item 特徴点に基づく確率的可視性制約とCBFの適用:各エージェントが観測する特徴点に基づき,その可視性を確率的に評価した上で,制御バリア関数(CBF)に組み込み,常時高い確率で共有視野が確保されるよう制御入力を設計する.
    \item 非ホロノミックなドローンダイナミクスへの対応と分散最適化:機体の並進および回転運動を同時に考慮するSE(3)上の非ホロノミックドローンモデルに対して,高次制約も扱える制御バリア関数(HOCBF)を導入し,各エージェントが局所的な情報交換を通じて分散最適化アルゴリズムにより制御解を求める枠組みを提案する.
\end{enumerate}

以下,第2章では制御バリア関数の基礎概念について説明し,第3章では問題設定を行う.第4章では共有視野のためのCBFを導入し,第5章では二次系システムのための高次CBFを提案する.第6章ではシミュレーション結果を示し,第7章で結論を述べる.
