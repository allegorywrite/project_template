\section{CBFs for Shared Field of View}

本章では,共有視野を保証するための制御バリア関数(CBF)を設計する.まず単一の特徴点を追従する場合の安全制約を導出し,次に複数の特徴点を追従する場合,さらに複数のエージェントが共通の特徴点を追従する場合へと拡張する.最後に,分散型実装について述べる.

\subsection{単一特徴点追従の安全制約}

単一の特徴点を追従する安全制約については,Trimarchiら\cite{Trimarchi2025}の研究に大きく影響を受けている.

エージェント位置を$T_i=(p_i, R_i)\in \mathrm{SE}(3), i\in\mathcal{A}$,環境内の特徴点を$q_l\in \mathcal{L}$とする.エージェント$i$の観測している特徴点集合を$\mathcal{C}_i$とする.$q_l\in\mathcal{C}_i$である条件は,\Eqref{eq:fov_condition}で示したように以下のように表される.
\begin{equation}
\begin{aligned}
\beta_l^{\top}(p_i)R_ie_c-\cos\Psi_\mathcal{F}>0
\label{eq:fov_condition_single}
\end{aligned}
\end{equation}

この条件に基づいて,安全集合を以下のように定義する.
\begin{equation}
\begin{aligned}
B_{i} = \beta_l^{\top}(p_i)R_ie_c-\cos\Psi_\mathcal{F}
\label{eq:safe_set_single}
\end{aligned}
\end{equation}

安全集合$B_i$の時間微分は以下のように計算される.
\begin{equation}
\begin{aligned}
\dot{B}_{i} &= \langle \mathrm{grad}_R\:B_i,\omega_i\rangle + \langle \mathrm{grad}_p\:B_i,v_i\rangle \\
&= -\beta_l^\top(p_i) R_i [e_c]_\times\omega_i - \frac{e_c^\top R_i^\top P_{\beta_l}}{d_{i,l}}v_i
\label{eq:safe_set_derivative_single}
\end{aligned}
\end{equation}
ここで,$d_{i,l}={\|q_l-p_i\|}$は特徴点$q_l$とエージェント$i$の距離,$P_{\beta_l} = I-\beta_l\beta_l^\top$は$\beta_l$に直交する平面への投影行列である.また,$\mathrm{grad}_R\:B_i$と$\mathrm{grad}_p\:B_i$はそれぞれ$B_i$の$R_i$と$p_i$に関する勾配である.

CBFの条件より,安全制約は以下のように表される.
\begin{equation}
\begin{aligned}
-\beta_l^\top(p_i) R_i [e_c]_\times\omega_i - \frac{e_c^\top R_i^\top P_{\beta_l}}{d_{i,l}}v_i \leq \gamma (\beta_l^{\top}(p_i)R_ie_c-\cos\Psi_\mathcal{F})
\label{eq:cbf_constraint_single}
\end{aligned}
\end{equation}
ここで,$\gamma > 0$はCBFのゲインパラメータである.

CBF制約を満たしつつ,目標位置$p_i^d$に追従するためのQPは以下のように定式化される.
\begin{equation}
\begin{aligned}
\min_{\xi_i} \: (p^d_{i}-p_{i,{k}}-hR_{i,k}v_{i,k})^\top Q_1 & (p^d_{i}-p_{i,{k}}-hR_{i,k}v_{i,k})
+ 
\begin{bmatrix}
\omega_{i,k}\\v_{i,k}
\end{bmatrix}^\top Q_2
\begin{bmatrix}
\omega_{i,k}\\v_{i,k}
\end{bmatrix} \\
\mathrm{s.t.} \quad
\beta_l^\top(p_i) R_i [e_c]_\times\omega_i &+ \frac{e_c^\top R_i^\top P_{\beta_l}}{d_{i,l}}v_i \leq \gamma (\beta_l^{\top}(p_i)R_ie_c-\cos\Psi_\mathcal{F})
\label{eq:qp_single}
\end{aligned}
\end{equation}
ここで,$Q_1$と$Q_2$は正定値重み行列である.

この最適化問題を一般的な形式に変換すると,以下のようになる.
\begin{equation}
\begin{aligned}
&\min_{\xi_i}\:J_i \\
J_i &= \frac{1}{2}\xi_{i,k}^\top H_i \xi_{i,k} + f_i^\top \xi_{i,k} \\
\xi_{i,k} &= \begin{bmatrix}
\omega_{i,k}\\v_{i,k}
\end{bmatrix}, \quad
H_i = 2\begin{bmatrix}
Q_{2,\omega} & Q_{2,\omega v} \\ 
Q^\top_{2,\omega v} & Q_{2,v}+h^2R_{i,k}^\top Q_1R_{i,k}
\end{bmatrix}, \\
f_i &= \begin{bmatrix}
0 \\ -2hR_i^\top Q_1 e_i
\end{bmatrix}, \quad e_i = p^d_{i}-p_{i,k} \\
\mathrm{s.t.} &\quad
\begin{bmatrix}
\beta_l^\top(p_i) R_i [e_c]_\times \\
\frac{e_c^\top R_i^\top P_{\beta_l}}{d_{i,l}}
\end{bmatrix}^\top \xi_{i,k} \leq
\gamma (\beta_l^{\top}(p_i)R_ie_c-\cos\Psi_\mathcal{F})
\label{eq:qp_single_general}
\end{aligned}
\end{equation}

\subsection{複数特徴点追従の安全制約}

複数の特徴点を追従することを考える場合,単純に各特徴点に対する安全集合の積集合を考えることもできるが,それでは安全集合が微分不可能になる可能性がある.そこで,確率的なアプローチを導入する.

特徴点$q_l \in \mathcal{L}$によってエージェント$i$における推定が成り立っている確率を以下のように定義する.
\begin{equation}
\begin{aligned}
\phi_{i}^l = P(p_i,R_i,q_l)
\label{eq:probability_single}
\end{aligned}
\end{equation}

この確率を用いて,新しい安全集合を以下のように定義する.
\begin{equation}
\begin{aligned}
B_{i} = 1-q-\prod_{l\in\mathcal{L}}(1-\phi_{i}^l)
\label{eq:safe_set_multi}
\end{aligned}
\end{equation}
ここで,$q \in (0,1)$はパラメータであり,環境内の特徴点$q_l \in \mathcal{L}$によってエージェント$i$における推定が達成される確率を$q$以上に制限することを意味する.

確率$\phi_{i}^l$は以下のように定義する.
\begin{equation}
\begin{aligned}
P(p_i,R_i,q_l) &= \left\{ \begin{array}{ll}
P_i^l & \mathrm{if} \quad q_l\in\mathcal{C}_i \\
0 & \mathrm{if} \quad q_l\in \mathcal{L}\setminus\mathcal{C}_i
\end{array} \right. \\
\mathrm{where} \quad P_i^l &= \frac{\beta_l^\top(p_i) R_i e_c -\cos\Psi_\mathcal{F}}{1-\cos\Psi_\mathcal{F}}
\label{eq:probability_definition}
\end{aligned}
\end{equation}

安全制約は以下のように表される.
\begin{equation}
\begin{aligned}
B_{i} > 0 \quad \forall i \in \mathcal{A}
\label{eq:safe_constraint_multi}
\end{aligned}
\end{equation}

安全集合$B_i$の時間微分は以下のように計算される.
\begin{equation}
\begin{aligned}
B_{i} &= 1-q-\eta_{i} \\
\eta_{i} &= \prod_{l\in\mathcal{L}}(1-\phi_{i}^l) \\
\dot{B}_{i} &= -\dot{\eta}_{i} \\
&= -\frac{d}{dt}\prod_{l\in\mathcal{L}}(1-\phi_{i}^l) \\
&= \sum_{l\in \mathcal{L}}(\prod_{k\neq l}(1-\phi_{i}^k))\dot{\phi}^l_{i} \\
\dot{\phi}^l_{i} &= \left\{ \begin{array}{ll}
\dot{P}_i^l & \mathrm{if} \quad q_l\in\mathcal{C}_i \\
0 & \mathrm{if} \quad q_l\in \mathcal{L}\setminus\mathcal{C}_i
\end{array} \right.
\label{eq:safe_set_derivative_multi}
\end{aligned}
\end{equation}

$P_i^l$の微分は以下のように計算できる.
\begin{equation}
\begin{aligned}
P_i^l &= \frac{\beta_l^\top(p_i) R_i e_c -\cos\Psi_\mathcal{F}}{1-\cos\Psi_\mathcal{F}} \\
\dot{P}_i^l &= \langle \mathrm{grad}\:P_i^l, \xi_W\rangle \\
&= \left\langle 
\begin{bmatrix}\mathrm{grad}_R\:P_i^l\\\mathrm{grad}_p\:P_i^l
\end{bmatrix},
\begin{bmatrix}\omega_i\\v_i
\end{bmatrix}
\right\rangle \\
\mathrm{grad}_p\:P_i^l &= \frac{1}{1-\cos \Psi_\mathcal{F}}\left(-\frac{e_c^\top R_i^\top P_{\beta_l}}{d_{i,l}}\right) \\
\mathrm{grad}_R\:P_i^l &= \frac{1}{1-\cos \Psi_\mathcal{F}}(-\beta_l^\top(p_i) R_i [e_c]_\times)
\label{eq:probability_derivative}
\end{aligned}
\end{equation}

これらから,複数の特徴点を追従するためのCBF制約は以下のようになる.
\begin{equation}
\begin{aligned}
\sum_{l\in \mathcal{L}\cap\mathcal{C}_i}(\prod_{k\neq l}(1-\phi_{i}^k)) \langle \mathrm{grad}_R\:P_i^l,\omega_i\rangle &+ \sum_{l\in \mathcal{L}\cap\mathcal{C}_i}(\prod_{k\neq l}(1-\phi_{i}^k)) \langle \mathrm{grad}_p\:P_i^l,v_i\rangle \\
&\geq -\gamma_0 (1-q-\prod_{l\in\mathcal{L}}(1-\phi_{i}^l))
\label{eq:cbf_constraint_multi}
\end{aligned}
\end{equation}
ここで,$\gamma_0 > 0$はCBFのゲインパラメータである.

CBF制約を満たしつつ,目標位置$p_i^d$に追従するためのQPは以下のように定式化される.
\begin{equation}
\begin{aligned}
\min_{\xi_i} \: (p^d_{i}-p_{i,{k+1}})^\top Q_1 & (p^d_{i}-p_{i,{k+1}})
+ \xi_{i,k}^\top Q_2 \xi_{i,k} \\
\mathrm{s.t.} \quad
\sum_{l\in \mathcal{L}\cap\mathcal{C}_i}(\prod_{k\neq l}(1-\phi_{i}^k)) \frac{\beta_l^\top(p_i) R_i [e_c]_\times}{1-\cos \Psi_\mathcal{F}}\omega_i &+ \sum_{l\in \mathcal{L}\cap\mathcal{C}_i}(\prod_{k\neq l}(1-\phi_{i}^k)) \frac{e_c^\top R_i^\top P_{\beta_l}}{(1-\cos \Psi_\mathcal{F})d_{i,l}}v_i \\
&\leq \gamma_0 (1-q-\prod_{l\in\mathcal{L}}(1-\phi_{i}^k))
\label{eq:qp_multi}
\end{aligned}
\end{equation}

一般的なQPの形式に変換すると,以下のようになる.
\begin{equation}
\begin{aligned}
&\min_{\xi_i}\:J_i \\
J_i &= \frac{1}{2}\xi_{i,k}^\top H_i \xi_{i,k} + f_i^\top \xi_{i,k} \\
\xi_{i,k} &= \begin{bmatrix}
\omega_{i,k}\\v_{i,k}
\end{bmatrix}, \quad
H_i = 2\begin{bmatrix}
Q_{2,\omega} & Q_{2,\omega v} \\ 
Q^\top_{2,\omega v} & Q_{2,v}+h^2R_{i,k}^\top Q_1R_{i,k}
\end{bmatrix}, \\
f_i &= \begin{bmatrix}
0 \\ -2hR_i^\top Q_1 e_i
\end{bmatrix}, \quad e_i = p^d_{i}-p_{i,k} \\
\mathrm{s.t.} &\quad
\begin{bmatrix}
\sum_{l\in \mathcal{L}\cap\mathcal{C}_i}(\prod_{k\neq l}(1-\phi_{i}^k)) \frac{\beta_l^\top(p_i) R_i [e_c]_\times}{1-\cos \Psi_\mathcal{F}} \\
\sum_{l\in \mathcal{L}\cap\mathcal{C}_i}(\prod_{k\neq l}(1-\phi_{i}^k))\frac{e_c^\top R_i^\top P_{\beta_l}}{(1-\cos \Psi_\mathcal{F})d_{i,l}}
\end{bmatrix}^\top \xi_{i,k} \leq
\gamma_0(1-q-\prod_{l\in \mathcal{L}}(1-\phi_{i}^k))
\label{eq:qp_multi_general}
\end{aligned}
\end{equation}

\subsection{複数エージェントの共通特徴点追従}

次に,複数のエージェントが共通の特徴点を追従する場合を考える.エージェント$i$と$j$が特徴点$q_l$を共有視野内に持つ条件は,\Eqref{eq:shared_fov_condition}で示したように以下のように表される.
\begin{equation}
\begin{aligned}
(\beta_l^{\top}(p_i)R_ie_c-\cos\Psi_\mathcal{F})(\beta_l^{\top}(p_j)R_je_c-\cos\Psi_\mathcal{F}) > 0
\label{eq:shared_fov_condition_repeat}
\end{aligned}
\end{equation}

特徴点$q_l \in \mathcal{L}$によってエッジ$(i,j) \in \mathcal{E}$における推定が成り立っている確率を以下のように定義する.
\begin{equation}
\begin{aligned}
\phi_{ij}^l = P(p_i,R_i,p_j,R_j,q_l)
\label{eq:probability_edge}
\end{aligned}
\end{equation}

この確率を用いて,新しい安全集合を以下のように定義する.
\begin{equation}
\begin{aligned}
B_{ij} = 1-q-\prod_{l\in\mathcal{L}}(1-\phi_{ij}^l)
\label{eq:safe_set_edge}
\end{aligned}
\end{equation}
ここで,$q \in (0,1)$はパラメータであり,環境内の特徴点$q_l \in \mathcal{L}$によってエッジ$(i,j) \in \mathcal{E}$における推定が達成される確率を$q$以上に制限することを意味する.

確率$\phi_{ij}^l$は以下のように定義する.
\begin{equation}
\begin{aligned}
P(p_i,R_i,p_j,R_j,q_l) &= \left\{ \begin{array}{ll}
P_i^l P_j^l & \mathrm{if} \quad q_l\in\mathcal{C}_i \cap \mathcal{C}_j \\
0 & \mathrm{if} \quad q_l\in \mathcal{L}\setminus(\mathcal{C}_i \cap \mathcal{C}_j)
\end{array} \right. \\
\mathrm{where} \quad P_i^l &= \frac{\beta_l^\top(p_i) R_i e_c -\cos\Psi_\mathcal{F}}{1-\cos\Psi_\mathcal{F}}
\label{eq:probability_edge_definition}
\end{aligned}
\end{equation}

安全制約は以下のように表される.
\begin{equation}
\begin{aligned}
B_{ij} > 0 \quad \forall (i,j) \in \mathcal{E}
\label{eq:safe_constraint_edge}
\end{aligned}
\end{equation}

安全集合$B_{ij}$の時間微分は以下のように計算される.
\begin{equation}
\begin{aligned}
B_{ij} &= 1-q-\eta_{ij} \\
\eta_{ij} &= \prod_{l\in\mathcal{L}}(1-\phi_{ij}^l) \\
\dot{B}_{ij} &= -\dot{\eta}_{ij} \\
&= -\frac{d}{dt}\prod_{l\in\mathcal{L}}(1-\phi_{ij}^l) \\
&= \sum_{l\in \mathcal{L}}(\prod_{k\neq l}(1-\phi_{ij}^k))\dot{\phi}^l_{ij} \\
\dot{\phi}^l_{ij} &= \left\{ \begin{array}{ll}
\dot{P}_i^l P_j^l + P_i^l \dot{P}_j^l & \mathrm{if} \quad q_l\in\mathcal{C}_i \cap \mathcal{C}_j \\
0 & \mathrm{if} \quad q_l\in \mathcal{L}\setminus(\mathcal{C}_i \cap \mathcal{C}_j)
\end{array} \right.
\label{eq:safe_set_derivative_edge}
\end{aligned}
\end{equation}

エージェントごとの制御入力について分解すると,以下のようになる.
\begin{equation}
\begin{aligned}
\dot{B}_{ij} &= -\dot{\eta}_{ij} \\
&= -\frac{d}{dt}\prod_{l\in\mathcal{L}}(1-\phi_{ij}^l) \\
&= \sum_{l\in \mathcal{L}}(\prod_{k\neq l}(1-\phi_{ij}^k))\dot{\phi}^l_{ij} \\
&= \sum_{l\in \mathcal{L}\cap\mathcal{C}_i \cap \mathcal{C}_j}(\prod_{k\neq l}(1-\phi_{ij}^k))P_j^l\dot{P}_i^l + \sum_{l\in \mathcal{L}\cap\mathcal{C}_i \cap \mathcal{C}_j}(\prod_{k\neq l}(1-\phi_{ij}^k))P_i^l\dot{P}_j^l
\label{eq:safe_set_derivative_edge_decomposed}
\end{aligned}
\end{equation}

これらから,複数のエージェントが共通の特徴点を追従するためのCBF制約は以下のようになる.
\begin{equation}
\begin{aligned}
&\sum_{l\in \mathcal{L}\cap\mathcal{C}_i \cap \mathcal{C}_j}(\prod_{k\neq l}(1-\phi_{ij}^k))P_j^l \langle \mathrm{grad}_R\:P_i^l,\omega_i\rangle + \sum_{l\in \mathcal{L}\cap\mathcal{C}_i \cap \mathcal{C}_j}(\prod_{k\neq l}(1-\phi_{ij}^k))P_i^l \langle \mathrm{grad}_R\:P_j^l,\omega_j \rangle \\
&+ \sum_{l\in \mathcal{L}\cap\mathcal{C}_i \cap \mathcal{C}_j}(\prod_{k\neq l}(1-\phi_{ij}^k))P_j^l \langle \mathrm{grad}_p\:P_i^l,v_i\rangle + \sum_{l\in \mathcal{L}\cap\mathcal{C}_i \cap \mathcal{C}_j}(\prod_{k\neq l}(1-\phi_{ij}^k))P_i^l \langle \mathrm{grad}_p\:P_j^l, v_j\rangle \\
&\geq -\gamma_0 (1-q-\prod_{l\in\mathcal{L}}(1-\phi_{ij}^l))
\label{eq:cbf_constraint_edge}
\end{aligned}
\end{equation}
ここで,$\gamma_0 > 0$はCBFのゲインパラメータである.

CBF制約を満たしつつ,目標位置$p_i^d$と$p_j^d$に追従するためのQPは以下のように定式化される.
\begin{equation}
\begin{aligned}
\min_{\xi_i, \xi_j} \: \sum_{i,j}(p^d_{i}-p_{i,{k+1}}-hR_{i,k}v_{i,k})^\top Q_1 & (p^d_{i}-p_{i,{k+1}}-hR_{i,k}v_{i,k})
+ 
\begin{bmatrix}
\omega_{i,k}\\v_{i,k}
\end{bmatrix}^\top Q_2
\begin{bmatrix}
\omega_{i,k}\\v_{i,k}
\end{bmatrix} \\
\mathrm{s.t.} \quad
&\sum_{l\in \mathcal{L}\cap\mathcal{C}_i \cap \mathcal{C}_j}(\prod_{k\neq l}(1-\phi_{ij}^k)) \frac{P_j^l\beta_l^\top(p_i) R_i [e_c]_\times}{1-\cos \Psi_\mathcal{F}}\omega_i \\
&+ \sum_{l\in \mathcal{L}\cap\mathcal{C}_i \cap \mathcal{C}_j}(\prod_{k\neq l}(1-\phi_{ij}^k)) 
\frac{P_i^l\beta_l^\top(p_j) R_j [e_c]_\times}{1-\cos \Psi_\mathcal{F}}\omega_j \\
&+ \sum_{l\in \mathcal{L}\cap\mathcal{C}_i \cap \mathcal{C}_j}(\prod_{k\neq l}(1-\phi_{ij}^k))\frac{P_j^le_c^\top R_i^\top P_{\beta_l}}{(1-\cos \Psi_\mathcal{F})d_{i,l}}v_i \\
&+ \sum_{l\in \mathcal{L}\cap\mathcal{C}_i \cap \mathcal{C}_j}(\prod_{k\neq l}(1-\phi_{ij}^k))\frac{P_i^le_c^\top R_j^\top P_{\beta_l}}{(1-\cos \Psi_\mathcal{F})d_{j,l}}v_j \\
&\leq \gamma_0 (1-q-\prod_{l\in\mathcal{L}}(1-\phi_{ij}^l))
\label{eq:qp_edge}
\end{aligned}
\end{equation}

一般的なQPの形式に変換すると,以下のようになる.
\begin{equation}
\begin{aligned}
&\min_{\xi_i, \xi_j} \: \sum_{i,j}J_i \\
J_i &= \frac{1}{2}\xi_{i,k}^\top H_i\xi_{i,k} + f_i^\top\xi_{i,k} \\
\xi_{i,k} &= \begin{bmatrix}
\omega_{i,k}\\v_{i,k}
\end{bmatrix}, \quad
H_i = 2\begin{bmatrix}
Q_{2,\omega} & Q_{2,\omega v} \\ 
Q^\top_{2,\omega v} & Q_{2,v}+h^2R_{i,k}^\top Q_1R_{i,k}
\end{bmatrix}, \\
f_i &= \begin{bmatrix}
0 \\ -2hR_i^\top Q_1 e_i
\end{bmatrix}, \quad e_i = p^d_{i}-p_{i,k} \\
\mathrm{s.t.} &\quad \begin{bmatrix}
\alpha_\omega & \alpha_v & \beta_\omega & \beta_v
\end{bmatrix}
\begin{bmatrix}
\omega_{i,k} \\ v_{i,k} \\ \omega_{j,k} \\ v_{j,k}
\end{bmatrix} \leq
\gamma_0\gamma
\label{eq:qp_edge_general}
\end{aligned}
\end{equation}
ここで,係数は以下のように定義される.
\begin{equation}
\begin{aligned}
\alpha_\omega &= \sum_{l\in \mathcal{L}\cap\mathcal{C}_i \cap \mathcal{C}_j}(\prod_{k\neq l}(1-\phi_{ij}^k)) \frac{P_j^l\beta_l^\top(p_i) R_i [e_c]_\times}{1-\cos \Psi_\mathcal{F}} \in \mathbb{R}^3 \\
\beta_\omega &= \sum_{l\in \mathcal{L}\cap\mathcal{C}_i \cap \mathcal{C}_j}(\prod_{k\neq l}(1-\phi_{ij}^k)) 
\frac{P_i^l\beta_l^\top(p_j) R_j [e_c]_\times}{1-\cos \Psi_\mathcal{F}} \in \mathbb{R}^3 \\
\alpha_v &= \sum_{l\in \mathcal{L}\cap\mathcal{C}_i \cap \mathcal{C}_j}(\prod_{k\neq l}(1-\phi_{ij}^k))\frac{P_j^le_c^\top R_i^\top P_{\beta_l}}{(1-\cos \Psi_\mathcal{F})d_{i,l}} \in \mathbb{R}^3 \\
\beta_v &= \sum_{l\in \mathcal{L}\cap\mathcal{C}_i \cap \mathcal{C}_j}(\prod_{k\neq l}(1-\phi_{ij}^k))\frac{P_i^le_c^\top R_j^\top P_{\beta_l}}{(1-\cos \Psi_\mathcal{F})d_{j,l}} \in \mathbb{R}^3 \\
\gamma &= 1-q-\prod_{l\in \mathcal{L}\cap\mathcal{C}_i \cap \mathcal{C}_j}(1-\phi_{ij}^l) \in \mathbb{R}
\label{eq:qp_edge_coefficients}
\end{aligned}
\end{equation}

\subsection{分散型実装}

不等式制約付き最適化問題を分散化する方法はいくつかあるが,本研究では主双対乗数法(Primal-Dual Method of Multipliers, PDMM)を用いる.PDMMは,不等式制約を処理する場合にADMMのようにスラック変数修正が必要なく,より効率的に解くことができる\cite{Zhang2024}.

IEQ-PDMM(inequality constraint primal-dual method of multipliers)を用いて上の式を分散化すると以下のようになる.
\begin{equation}
\begin{aligned}
\xi_i &= \underset{\xi_i}{\mathrm{argmin}} \:J_i(\xi_i)+z_{i|j}^\top A_{ij}\xi_i+\frac{c}{2}\|A_{ij}\xi_i-\frac{1}{2}\gamma_0\gamma\|^2 \\
y_{i|j} &= z_{i|j}+2c(A_{ij}\xi_{i}-\frac{1}{2}\gamma_0\gamma) \\
&\mathbf{node}_j\leftarrow \mathbf{node}_i(y_{i|j}) \\
&\mathbf{if}\:y_{i|j}+y_{j|i}>0 \\
&\qquad z_{i|j}=y_{j|i} \\
&\mathbf{else}\: \\
&\qquad z_{i|j}=-y_{i|j}
\label{eq:pdmm}
\end{aligned}
\end{equation}
ここで,$A_{ij} = \begin{bmatrix} \alpha_\omega \\ \alpha_v \end{bmatrix}$,$A_{ji} = \begin{bmatrix} \beta_\omega \\ \beta_v \end{bmatrix}$である.

QPの形式に表すと,以下のようになる.
\begin{equation}
\begin{aligned}
&\min_{\xi_i}\:\hat{J}_i = \frac{1}{2}\xi_{i,k}^\top \hat{H}_i\xi_{i,k} + \hat{f}_i^\top \xi_{i,k} \\
\xi_{i,k} &= \begin{bmatrix}
\omega_{i,k}\\v_{i,k}
\end{bmatrix}, \quad
\hat{H}_i = \begin{bmatrix}
2Q_{2,\omega}+c\alpha_\omega^\top \alpha_\omega & 2Q_{2,\omega v}+c\alpha_\omega^\top \alpha_v \\
2Q^\top_{2,\omega v}+c\alpha_v^\top \alpha_\omega & 2Q_{2,v}+2h^2R_{i,k}^\top Q_1R_{i,k}+c\alpha_v^\top \alpha_v
\end{bmatrix}, \\
\hat{f}_i &= \begin{bmatrix}
z_{i|j}\alpha_\omega^\top -\frac{c}{2}\gamma_0\gamma\alpha_\omega^\top \\
-2hR_i^\top Q_1 e_i+z_{i|j}\alpha_v^\top -\frac{c}{2}\gamma_0\gamma\alpha_v^\top
\end{bmatrix}, \quad e_i = p^d_{i}-p_{i,k}
\label{eq:pdmm_qp}
\end{aligned}
\end{equation}

この計算手法では制約式がすべてソフト制約化するため,最適化中に元の制約式を満たすことを保証できない.そこで,Tanら\cite{Tan2022}が提案したCBF-induced QPを用いることも考えられる.これは,ADMMベースのCBF分散化手法であり,最適化中に元の制約式を満たすことを保証する.

本章では,共有視野を保証するためのCBFを設計し,それに基づく制御則を提案した.次章では,二次系システムのための高次CBFを導入し,より実際的なドローンダイナミクスに対応する手法を提案する.
